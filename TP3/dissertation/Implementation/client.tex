\section{Client}

\subsection{Overview}

This section will analyse how appropriate our client design  was for the purpose of achieving the aims outlined in the client section.\footnote{See section x.y for a detailed discussion}  Some of our design choices worked well to achieve our aims, such as the interfaces used by the controller to separate concerns. Another decision that worked well was our use of the object oriented features of java to implement chat windows, which ended up reducing the problem of internal routing of messages to the correct windows. These aspects will be discussed in the 'Design Changes' section.

While implementing the client, it became clear that some of our larger design choices were naive, and further design work had to be conducted. One case involved the interaction between the controller and networking subsystem. These changes will be discussed in the 'Design' Changes section. In some cases, smaller design choices required larger changes. Of particular significance, our understanding of the practical working of the GIM protocol to create personal chats proved to be ill informed. This required design work reaching over the protocol, and the model component of the system. This process tested our ability to collaborate as a team to implement an interacting system. The degree of our success of our changes will be discussed in the 'Collaboration' section. Furthermore, as we became more familiar with the swing environment, our code went through evolutionary steps to implement certain features in a more sensible manner, such as how displaying updates to user information was handled. The 'Evolution of Code' section will discuss problems we identified with code, which was functional,  but deemed to be hard to maintain on reviewing our code, such as in the above case case, and the merits of our changes we made. 

\subsection{Design Changes}



\subsection{Collaboration}

The first draft of the GIM protocol treated all conversations as 'rooms', and did not distinguish group chats and personal chats.\footnote{internal reference to protocol section} In designing the protocol, we wished to keep it abstract and not fall into the trap of implicitely implementing features in the protcol that the internal workings of the client and server could handle. Our goal was not to develop a protocol suitable for only one type of implementation. Originally, it was believed that the client could distinguish personal chats and group chats by storing internal records of the type of outgoing invites to chat. In the case of being invited to chat, we believed using the protocol's "USER" argument in the "ROOM" command would be sufficient to count the amount of users in the "room" and determine the type of chat. However, as we began implementing room creation, it became clear that the initial group user list could be of size 1, and thus the wrong chat window could be created. As a result, the protocol engineer had to be consulted to make changes to the protocol, and the client ammended to reflect these changes. 

This change was a test of the boundary of responsibilites within our system, as it effected communication with the server, and the back end of the client. The protocol engineer's solution to the issue was to add a 'type' argument to the room command. In the case creating a room, the type now had to be specified. To allow a user to work out what room type a chat had, the 'type' argument would be used, with a room identifier. This kept the protocol abstract, as some conceivable implementations of the protocol may not be interested in this. In order to deal with these changes, our client had to be designed to sequence responses from the server with certain requests. As the protocol did not include sequence numbers, we had to re-design the model to perform this task. It was apparent that the amount of "talk" between the client and server to achieve the goal of starting a chat was now more complex, which required an understanding of what needed to be stored at each stage. This high level plan was determined:

Creation:
\begin{enumerate}
\item Client adds a type of the room to be created to the new room queue in the model.
\item Client notes a list of user(s) invited to chat in the invitations queue in the model.
\item Client sends request to server to create a room of this type.
\item Server responds with 'created' with the room id.
\item Client matches the type of chat with the 'new room' queue, and the user list from the 'invitations queue.'
\item Client spawns, or updates existing window in the window list with the room id to send messages to, and the user(s) in the chat.
\item The client is informed that the participant has joined the room.
\end{enumerate}

Invitation:
\begin{enumerate}
\item Client receives invite to chat from server on certain room id from user. Store username in a "invited" queue in model.  
\item Client asks server the type of the room that it has been invited to. 
\item Server responds with 'Personal' or 'Group'
\item Client matches request with response from the "invited" queue.
\item Client spawns the appropriate chat window, or updates the internal state information of an existing window with the new roomid in the window list. Information includes who is in the room, and the id to send messages to.
\item If it is a group chat, client asks server for the user list in the room
\end{enumerate}

In retrospect, our trade off between keeping an abstract protocol may have been too high. Aside from this communication between the client and the server, there are complexities within the client code to ensure these events are performed in sequence which left much opportunity for error. For example, a user should not be allowed to send a message before the conversation participant has also joined the room, in the case of a personal chat between steps 6 and 7. This meant that safegaurds had to be considered during these sequence of events to ensure messages were not dropped, while ensuring the user did not have to 'wait' for the other user to enter the room, before entering a message. This motivated the need for a boolean value indicating whether the chat participant was in the room, internal to chat windows. While this value was false, messages are buffered to be sent when it turned to 'true.' A further issue of synchronicity was internal to the code. As the GUI was running on a seperate thread from the incoming networking thread, it was conceivable that an incoming message could occur before the internal room id was updating in step 6. In fact, this was an issue that was a subtle issue that was not recognised until late in development, and we struggled to identify. In retrospect, part of our problem was not identifying where the swing event queue needed to be used (as outlined in the previous section), compounded by a complex set of events that needed to occur to establish a chat. Throughout this project, we became more aware of the careful approach required from threading. 

\subsection{Evolution of Code}
