\section{Requirements}

Our first task was to establish what format our application would take. We had two main choices which would dictate how the project would proceed:

\begin{enumerate}
\item{One-to-One} - A similar style to Windows Live Messenger or AIM where users have a contact list of friends and typical conversations take place between two clients. 
\item{Multicast}  - IRC is the most prominent example of multicast instant messaging. Users connect to a server then join a room within that server. Rooms can have many users and will persist regardless of active users.
\end{enumerate}

Within these basic frameworks, we also had the option to merge aspects from each, for example basing a one-to-one application on the IRC protocol.

Ultimately it was decided that we would create a one-to-one messenger program, with a consideration towards group conversations. 

\subsection{Conception of Features}

Once we had established that we were going to use the one-to-one paradigm, the nest task was to determine what we believed to be the important features of an instant messenger, and what was achievable within the scope of the project. This process involved several team meetings where we simply discussed our experience with a variety of programs and picked areas where we wished to draw from. Due to the popularity of instant messenger programs, they have undergone constant evolution, and continue to do so. Some of our work had already been done.

The constant iteration of instant messenger interfaces provides us with a solid foundation with which to base our client GUI. Our experience of these programs allowed us to highlight features which we felt were achievable and, more importantly, useful to users. We conceived a feature set split into 4 categories of importance using the MoSCoW method.

\subsubsection{Must Have}

\begin{itemize}
\item{Send Messages}
\item{Graphical User Interface}
\item{User Nicknames}
\item{Contact list}
\item{User Status}
\end{itemize}

\subsubsection{Should Have}

\begin{itemize}
\item{URL Parsing}
\item{Display Pictures}
\item{File Transfers}
\item{Personal Messages}
\item{Smilies}
\end{itemize}

\subsubsection{Could Have}

\begin{itemize}
\item{User Profile}
\item{Custom Commands}
\item{Themes}
\item{Plug-in Support}
\item{VoIP}
\end{itemize}

\subsubsection{Would Like To Have}

\begin{itemize}
\item{Contact List Grouping}
\item{Offline Messaging}
\item{Chat Logging}
\item{Custom Fonts and Colours}
\end{itemize}

This list was decided upon by taking into account what we believed to be each features' necessity, usefulness, and difficulty of implementation. The requirements in the 'Must Have' category were taken from the initial problem specification, and the other categories were decided using the criteria described previously.

The 'Must Have' features generally contain the basic elements of an instant messenger, such as sending messages and a graphical user interface. Sending messages, a GUI, and user statuses were taken from the problem definition, however we felt it was crucial and within the scope of this project to include contact lists and user nicknames as must have features.

Should Have features are those which we felt were within the scope of the project and would significantly enhance the users' experience. URL parsing is the ability for user to select hyperlinks in the chat window. This was given high priority due to our experience of using other IM clients, which often involves sending contacts links to various websites. Display pictures are images that a user uses to represent themselves with to their contacts. While display pictures do not directly impact the functionality of the program, we felt that they would make the chatting experience more personal and users may expect to see what has been a standard feature of similar programs for some time. We considered the ability to send files between users to be a useful feature but were are that it would potentially be one of the most difficult items on the list to implement. Personal messages are one of the easier features on the list. A personal message is a small message a user sets on the interface that all other users can see, typically underneath the username and given less prominence. As this was considered to be simple to implement, it was assigned a relatively high priority. Smilies (also known as emoticons) are small icons used to represent emotions in chat. While they add visual appeal, smilies would not add significant functionality as text-based representations can be used.

Many of the could have features involve customisation of the interface. User profiles are pages in the interface which would contain details on that user which can be viewed by contacts. Beyond the basic concept we had not decided how these would function or what they would contain. Custom commands would act in a similar way to an IRC client; commands such as '/whois' or '/join'. As these were custom commands, the user would be able to set their own keywords for a set of possible commands. Themes are another feature to further customisation. A theme is a preset of colours used to alter the interfaces' look and feel. Two of the more demanding requirements we conceived were "Plug-in Support" and VoIP. This is the ability for other developers to create additional features which can be slotted in to certain areas, for example support for other messenger protocols. VoIP (Voice over Internet Protocol) is a protocol designed to accommodate voice chat between clients. While this is a very desirable feature it was generally considered to be out with the scope of the project, but we decided to include it so that a decision could be made during the implementation phase depending on our progress.

Features which we would like to have are those which are low priority, but somewhat desirable. Contact list grouping is where the user is able to create subsets of their contacts to make managing large contact lists easier. This would likely be included in the form of displaying Online or Offline contacts, however this feature represents capability beyond that basic functionality, such as user-defined custom groups. Offline messaging allows users to send messages to each other regardless of their status. When a user logs into the system, they receive any messages that were sent to them while they were not logged in. This is useful for short disconnects to prevent messages being lost. Chat logging is the ability to store conversations between users, while this feature can be useful, it is generally not used often by users, hence it's low priority. Custom fonts and colours are related to themes, but would provide a deeper level of customisation. Themes were given a higher priority because they were considered easier to implement and would provide an adequate level of customisation for most users.


\subsubsection{Rejected Features}

We decided to aim the application at users who are comfortable with general computer use. As a result there were features which we agreed were not desirable or worth our time implementing.

\begin{itemize}

\item{Nudges/Winks}
\item{Audio Cues}
\item{Spell Checking}
\item{URL Warnings}

\end{itemize}

Nudges are events which one user sends to another to force their chat window into focus. These were considered by many to be an irritation which is prone to abuse, hence our decision to not include them. Winks are a similar feature with more multi-media. Spell Checking was considered, but due to the typically informal nature of IM conversations, it was not included. URL warnings are quick pop-ups that appear whenever a user clicks on a hyperlink to warn them that content may be harmful. As we has made the assumption that users will have a basic level of competence with computers, this feature was not included.
