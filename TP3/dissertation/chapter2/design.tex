\chapter{Design}
\label{design}

\newcommand\SLASH{\char`\\}

\section{Protocol Design}

\subsection{Protocol Overview}

\begin{figure}[!h]
    \begin{center}
        \includegraphics[scale=0.65]{chapter2/diagrams/protocol_high_level.png}
        \caption{The exchange of messages between a client and the server}
        \label{highLevelDia}
    \end{center}
\end{figure}

Gim uses a client-server architecture, where one computer (known as the Sever) acts as a central point to which other computers (the Clients) connect. The clients do not communicate directly with each other and all communication takes place between the clients and the server. If a client wishes to send a message to another client it must first got through the server.

In GIM, the Protocol is responsible for enabling communication between the clients and the server in a reliable and consistent manner. A protocol is a set of rules that determine the format and transmission of data between computers. The syntax (the structure, or format) and semantics (the meaning) of the Protocol are discussed in the next section.

At the highest level of abstraction, the GIM Protocol works in a very simple manner.  A client connects to the sever and they exchange messages until the connection is closed, as shown in Figure ~\ref{highLevelDia}.

In practice there are several clients connected to the server at once, however as the clients do not directly interact with each other and do not need to know about each other, the entire system can be described as above.

\subsection{Protocol Specification}
\subsubsection{Syntax}

The GIM Protocol uses a simple, text based syntax. Each full command begins with a colon followed by a command name and its arguments. A second colon marks the end of the headers and beginning of the data segment. A semi-colon marks the end of the data segment and the termination of the command. The basic structure of a full command is shown below:

\texttt{:<COMMAND\_NAME> <ARGUMENTS>: <DATA>;}

Commands names are predefined and \texttt{<COMMAND\_NAME>} can be any 1 of the following 19 values:

\texttt{
\begin{tabular}{ | p{3.3cm} |p{3.3cm}  |p{3.3cm} |p{3.3cm} | } 
    \hline
    AUTH & FRIENDREQUEST & MESSAGE & ROOM \\
    \hline
    BROADCAST & GET & OKAY & SERVERSTATUS \\
    \hline
    ERROR & INFO &PING & SET \\
    \hline
    FRIEND & KILL & PONG & UPDATE \\
    \hline
    FRIENDLIST & LOGOUT & QUIT & \\
    \hline
 \end{tabular}
 }
 
\texttt{<ARGUMENTS>} is a set of 0 or more predefined arguments which can be passed to the command in order to change its behaviour. Each command has its own set of arguments, some commands require arguments, some commands accept more than one argument and some commands have no arguments. The protocol specification defines over 60 arguments and a full listing for each command can be found in the \emph{GIM Protocol Specification Document}.

Unlike the command and arguments segments, the \texttt{<DATA>} segment does not have predefined values and has a varying syntax across different commands. The majority of the data is likely to have been provided by the user at some point, and as such the data segment is considered to be ``unsafe" and is encoded (See \emph{Encoding, Limits and Restrictions} for information about how the data is encoded). A full specification for the data segment of each command can be found in the \emph{GIM Protocol Specification Document}.

For example, the following is an example of a command sent from the client to the server requesting the Nickname and Status of the users joe@example.com and bill@gmail.org:

\texttt{:GET NICKNAME STATUS: joe\SLASH U+0040example.com bill\SLASH U+0040mail.org;}

To which the server may reply:

\texttt{:INFO NICKNAME STATUS: joe\SLASH U+0040example.com\\
Jeo\\
ONLINE\\
bill\SLASH U+0040mail.org\\
Billy The Kid\\
AWAY;}

\emph{(Please note that the data segments is the previous examples have been encoded as is defined below in Encoding, Limits and Restrictions.)}

\subsubsection{Roles and States}
The protocol defines two separate roles: The Client role and the Server role. Each role is only able to send a specific subset of commands but must also be able to receive and understand all of the commands sent by the opposite role. This means that both the client and server must implement all of the commands in some way.

Having clearly defined roles is an integral part of the GIM Protocol. This allows the semantics of a command to be used to generate a response in cases where both roles are able to send and receive a command but where the command has a different meaning depending on its source. 

For example, the \texttt{AUTH} command can be sent by both the client and the server roles. When sent by the server it is used to indicate the current state of the client (Either Logged-in or Unauthorized) but when sent by the client it indicates that they wish to login or register a new account. This means that the same command must be implemented differently depending on the role of the sender.

As mentioned above, the client role has 2 states: Logged-in and Unauthorized. When the client first establishes a connection it is placed in the unauthorized state. This means that it has access to an even smaller subset of commands, only those essential to logging-in or registering a new account. Once the client has successfully logged-in it is granted access to the full set of client commands.

\subsubsection{Security and Encryption}

\subsubsection{Encoding, Limits and Restrictions}

\subsection{Protocol Evolution}


\section{Client Evaluation}
\label{client_eval}




This section will detail the process of designing the GIM client; the application used the communicate with the server. Primarily this will concern the Graphical User Interface (GUI) and the feature set.

\section{Conception of Features}

One of the first tasks for the project was to determine what we beleived to be the important features of an instant messenger, and what was acheivable within the scope of the project. This process involved several team meetings where we simply discussed our experience with a variety of programs and picked areas where we wished to draw from. Due to the popularity of instant messenger programs, they have undergone constant evolution, and continue to do so. Some of our work had already been done.

\newline

The constant iteration of instant messenger interfaces provides us with a solid foundation with which to base our client GUI. Our experience of these programs allowed us to highlight features which we felt were acheivable and, more importantly, useful to users. We conceived a feature set split into 4 categories of importance using the MoSCoW method.

\subsubsection{Must Have}

\begin{itemize}

\item{Send Messages}
\item{Graphical User Interface}
\item{User Nicknames}
\item{Contact list}
\item{User Status}

\end{itemize}

\subsubsection{Should Have}

\begin{itemize}

\item{URL Parsing}
\item{Display Pictures}
\item{File Transfers}
\item{Personal Messages}
\item{Smilies}

\end{itemize}

\subsubsection{Could Have}

\begin{itemize}

\item{User Profile}
\item{Custom Commands}
\item{Themes}
\item{Plug-in Support}
\item{VoIP}

\end{itemize}

\subsubsection{Would Like To Have}

\begin{itemize}

\item{Contact List Grouping}
\item{Offline Messaging}
\item{Chat Logging}
\item{Custom Fonts and Colours}

\end{itemize}

This list was decided upon by taking into account what we belived to be each features' necessity, usefulness, and diffiulty of implementation. The requirements in the "Must Have" category were taken from the initial problem specification, and the other categories were decided using the criteria described previously.

The "Must Have" features generally contain the basic elements of an instant messenger, such as sending messages and a graphical user interface. Of note is the inclusion of user statues; this was included here as at a basic level, status would simply indicate whether a user was online or not. The final application also supports "Away" and "Busy", but these are less important than "Online" or "Offline" as these have implications beyond what the end user will see.

\newline

Should Have features are those which we felt were within the scope of the project and would significantly enhance the users' experience. URL parsing is the ability for user to select hyperlinks in the chat window. This was given high priority due to our experience of using other IM clients, which often involves sending contacts links to various websites. Display pictures are images that a user uses to represent themselves with to their contacts. While display pictures do not directly impact the functionality of the program, we felt that they would make the chatting experience more personal and users may expect to see what has been a standard feature of similar programs for some time. We considered the ability to send files between users to be a useful feature but were are that it would potentially be one of the most difficult items on the list to implement. Personal messages are one of the easier features on the list. A personal message is a small message a user sets on the interface that all other users can see, typically underneath the username and given less prominence. As this was considered to be simple to implement, it was assigned a relativly high priority. Smilies (also known as emoticons) are small icons used to represent emotions in chat. While they add visual appeal, smilies would not add significant functionality as text-based representations can be used.

\newline

Many of the could have features involve customisation of the interface. User profiles are pages in the interface which would contain details on that user which can be viewed by contacts. 

\section{Server Design}
\label{ServerDesign}

The GIM server is primarily responsible for synchronising the communication between the connected clients and keeping a persistent record of user information. In particular the server is responsible for:

\begin{itemize}
    \item{Authenticating users}
    \item{Routing chat messages for one client to one or more other clients}
    \item{Sending requests from one client to another client (such as friend requests or chat invites)}
    \item{Notifying subscribed users of changes to another user's information}
    \item{Storing user information such as nicknames, passwords, display pictures, friend lists, etc.}
\end{itemize}

This section discusses how the design of the server copes with these requirements and the the reasoning behind these decisions.

\subsection{Server Structure}

From the very beginning the server was designed to be secure, scalable, and simple. Users must trust it to keep their information safe and route messages correctly, and it must be able to cope with any number of connected clients.

In order to manage this, the server generates a new worker for every connected client, as shown in figure \ref{WorkersDia}. This new worker acts as the single point of contact for its respective client. This greatly simplifies the design of the server as we can treat each connection individually and allows the server to easily scale to a large number of clients.

\begin{figure}
    \begin{center}
        \includegraphics[scale=0.6]{Design/diagrams/server_workers.png}
        \caption{The GIM Server with 3 connected clients, each with their own worker.}
        \label{WorkersDia}
    \end{center}
\end{figure}

Each worker has two buffers, one which stores commands read from the network (the command buffer), and one which stores commands to be written to the network (the response buffer). The worker continually reads commands from the command buffer, processes them, and puts the responses into the response buffer. At the same time the worker is also performing network reads and writes to fill and empty the respective buffers. This is show in figure \ref{WorkerDatailedDia}.

The buffers allow the worker to run using several threads which increases the performance on multi-core systems as the server is able to read from the network, process commands, and write to the network in a truly concurrent fashion. This means that at any one time a worker could be handling up to three different commands, each at a different stage of execution, and still have several commands stored in the buffers waiting to be used.

\begin{figure}
    \begin{center}
        \includegraphics[scale=0.6]{Design/diagrams/worker_detail.png}
        \caption{The internal components of a single worker.}
        \label{WorkerDatailedDia}
    \end{center}
\end{figure}

\subsection{User Accounts}
A user is a registered account on the server with a unique identifier (an email address), password, nickname, personal message, display picture, status, and friend list. 

\begin{itemize}
\item{{\bf Identification}\\
The user's identifier and password are used to authenticate the user at login. These cannot be changed one the account has been created.} 

\item{{\bf Nickname}\\
A user's nickname is a short and familiar name for that user. A nickname does not need to be unique and is not intended to be used as an identifier by the system. By default a user's nickname is their email address.}

\item{{\bf Personal Message}\\
The personal message is a medium length message (usually only a sentence or two) which the user can use to share news or any information about themselves.}

\item{{\bf Status}\\
A user can have one of several statuses: Online, Away, Busy and Appear Offline. The first three statuses are used simply as an indication of the user's likelihood to respond to a message. The Appear Offline status allows the user to appear to other users as though they were not signed in but still receive updates and messages from other users.}

\item{{\bf State}\\
A user can have one of only two states, Online or Offline. States should not be confused with a user's status. A users status is simply an abstract representation, picked by the user, while a users state is an internal representation of the actual online state of the user. A user is only considered to be Online if they are connected using a client, and that client has logged-in successfully, otherwise they are considered to be Offline.}

\item{{\bf Friend List}\\
Each user has their own Friend List, a list of users who have accepted a friend request from the owner of the Friend List. When a user accepts a friend request, the sender and receiver are added to each other's Friend List.}
\end{itemize}



\subsection{Client-to-Client Messaging}
\label{c2c}
No direct client-to-client communication is possible. However, workers and (in an ad-hoc fashion) clients are able to indirectly communicate with other clients (i.e. clients to which they are not connected) by passing commands to the other client's worker. This is the server's main method of inter-client communication and relies heavily on the use of the buffers to pass messages from one worker to the network writer or command processor of another worker.

As the protocol defines no client to client communication is possible before a user has logged in, and as user can only be logged in on one client at a time, only a user id is necessary to send a command to a user. As long as a record is kept of which worker belongs to each online user, then sending a message to a user is no more complicated than identifying the particular user.

\subsection{Rooms and Message Routing}
\label{message_routing}

Message routing in the server is fairly simple because of the use of rooms. A room is a collection of one or more users and each room has a unique identifier. In order to join a room a user must either create a new room (in which case they are automatically placed into the room) or they must be invited to the room by a user already in the room. A room can be one of two types: personal or group. A personal room is much more restrictive than a group room. A personal room must be provided with the identifier of one other user at time of creation and is limited to only 2 users. Once a personal room has been created you cannot invite any other users. A group room has no such restrictions. Any number of users may be invited to a group room at any time.

All chat messages in GIM are sent to Rooms rather than particular users. The server must maintain a list of the users in each room, and upon receipt of a message for that room, forward it to all of the users in the room (excluding the sender of the message). This is done using the method described in section \ref{c2c}. A \texttt{:MESSAGE:} command is generated and placed into the Response Buffer of the worker assigned to each of the recipients ready for the worker to send to the client.

\subsection{Subscriptions, Privacy and Notifications}
One of the server's most critical responsibilities is notifying users of changes to other users. This is done using the concept of subscriptions. A user (the subscriber) is considered to be subscribed to another user (the subscribee) if they meet at least one of two conditions:

\begin{enumerate}
\item{The subscriber has the subscribee in their friend list, or had them in their friend list at some point in the past}
\item{Both the subscriber and subscribee are in the same room}
\end{enumerate}

When a user changes any piece of their user information such as their status, nickname, or personal message, the server needs to notify other clients so that they can update their information. The server generates a list of subscribers for this user and removes offline and blocked users. It then issues an \texttt{:UPDATE:} command to each of the clients, notifying them of the user and the piece of information which has been updated. Note that the server does not send the updated information, it merely notifies the client of the update, allowing it to request the updated information for the server when and if it needs it.

Subscriptions are also used as a method of access control. A user is only able to request information about another user if they are subscribed to that user. This limitation means that a user is effectively granting another user access to read their information by adding them to their friend list. However, the converse is not true. Removing a user from your friend list does not remove their access to your information as they may still have you in their friend list. In order to combat this, you are able to block a user which removes this access to your information and status updates. This enhances privacy and gives the user control of who is able to access their data.

\subsection{Synchronisation}
The server has a large section of backing data which stores all of the users, rooms, workers and any other operational data required by the server. Due to the highly concurrent nature of the server it is very important to ensure that the data is properly synchronised. This is to prevent race conditions where two writes occur on an object at the same time giving an unpredictable output, as show in \ref{RaceConditionDia}.

\begin{figure}
    \begin{center}
        \includegraphics[scale=0.6]{Design/diagrams/server_race_condition.png}
        \caption{Unsynchronised workers updating the same object at the same time, resulting in a race condition and giving the unexpected value of 6 rather than 7.}
        \label{RaceConditionDia}
    \end{center}
\end{figure}

To stop this the server must lock (claim ownership of) the data before it is updated. This is a very tricky thing to do as as the lock must be long enough to ensure that no updates occur between reading the original data and writing the updated data, while also keeping the lock as short as possible to avoid long periods of blocking and keep performance high.

The server must also make sure only to lock when needed, and not obtain a lock for objects it does not require. This simplest solution would be to lock the entire data object, and only allow reads and writes through a proxy. However this would be extremely costly in terms of performance as writes to individual objects inside the data object would block writes to other objects, meaning that at any one time only a single write could occur. Instead the data object must be clever and only lock the objects which are to be written to, allowing for simultaneous writes to separate objects while still stopping race conditions from occurring.

\begin{figure}
    \begin{center}
        \includegraphics[scale=0.6]{Design/diagrams/server_locking.png}
        \caption{Synchronised workers using locking to read and write the updated value, giving the expected value of 7.}
        \label{lockingDia}
    \end{center}
\end{figure}

\subsection{The Controller and Timeout Manager}
Even though the server should be capable of functioning autonomously, some of the features defined by the protocol require interaction with the server that deviates from the client-server model. For example, the \texttt{:BROADCAST:} command allows the server to send a message to all of the clients it is connected to. The controller allows authorised users to interact through a command line interface with the server. These authorised users are allowed by the operating system to connect to the session that is running the server. They are then able to execute several commands defined by the controller to broadcast messages and control the server. These commands are independent from protocol commands.

For example, the \texttt{:QUIT:} command tells the server to shut down and create a persistence file to store user data, and the following would send a \texttt{:BROADCAST:} command to all of the connected clients:

\texttt{BROADCAST We're shutting the server down in 15 minutes.}

The server needs to periodically check for clients which have not sent a command in the last 15 seconds (as defined in the protocol) and disconnect them. This means that the server must compare the time the last command was received by every worker with the current time. If the difference exceeds 15 seconds, they are disconnected.

Due to the nature of the Controller and Timeout Manager, they both must run on separate threads from the rest of the server or risk blocking a more important function, such as accepting incoming connections and creating workers for them.

\begin{landscape}
    \begin{figure}
        \begin{center}
            \includegraphics[scale=0.5]{Design/diagrams/server_uml.png}
            \caption{UML showing the structure of the GIM Server.}
            \label{umlDia}
        \end{center}
    \end{figure}
\end{landscape}


\section{Networking Design}

\subsection{Client Networking Structure}

We determined the role of the client networking component to be to maintain a connection with the server, read commands from the command buffer and pass them to the server, and pass commands from the server to the client. These two concerns came with different challenges; where keeping a connection with the server involved considering how to manage writing and reading data to the socket simultaneously, and sending and receiving commands involved considering how best to parse data coming from the socket, and pass data between components.

\subsubsection {Maintaining a Connection}

The connection to the server had to be designed such that our program could freely write to the socket, and receive commands with the appearance of concurrency. Furthermore, there client code should not have to worry about doing two simultaneous writes to the server, if there is any multithreaded aspects that could cause this. In reality, we would have to manage the connection to the socket so that the socket would not be written to while a read or write was in progress. 



\subsubsection {Command Handling }











