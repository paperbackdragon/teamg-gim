\section{Client Design}
In designing the client, we had the high level aim of creating a modular system to allow team members to take responsibility for certain aspects of the system. From an early stage in the design process, we were aware that the client had a large set of responsibilities. We determined these responsibilities to be;  to maintain a connection with the server, implement the GIM protocol, provide a user interface to the system and keep a record of up to date information about the user’s friends. Our first task was to design a set of interacting sub-components to handle these responsibilities.

It was agreed that the MVC (Model-Viewer-Controller) architectural pattern, widely used for applications involving a graphical user interface, was a useful model to base our discussions around. This model abstracts the UI (view) from the back end of the system. When a user performs an action, the controller updates a collection of data associated with the system, and updates the interface to reflect any changes. This seemed appropriate to our needs of keeping and displaying an up to record about the user’s friend list and creating an interface, and would allow us to split responsibilities between the back and front end of the GUI. 

However, we faced challenges in adapting an additional networking component into this framework (to implement the GIM protocol.) We had the choice to conceptually view it as either an additional interface, which ‘viewed’ changes to the model, or indirectly (by way of the network) the controller component. [wtf... discussion about merits of both] We decided it would be useful to treat the networking code as part of the controller, as it would be modifying the model based on the server’s response to its calls, and modifying the GUI to reflect these changes.
	
Our next challenge was to more formally design the roles of each component, and the boundaries between them. 
