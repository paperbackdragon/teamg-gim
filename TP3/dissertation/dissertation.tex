\documentclass{l3proj}
\usepackage{lscape}

\makeatletter
\newcommand\ackname{Acknowledgements}
\if@titlepage
  \newenvironment{acknowledgements}{%
      \titlepage
      \null\vfil
      \@beginparpenalty\@lowpenalty
      \begin{center}%
        \bfseries \ackname
        \@endparpenalty\@M
      \end{center}}%
     {\par\vfil\null\endtitlepage}
\else
  \newenvironment{acknowledgements}{%
      \if@twocolumn
        \section*{\abstractname}%
      \else
        \small
        \begin{center}%
          {\bfseries \ackname\vspace{-.5em}\vspace{\z@}}%
        \end{center}%
        \quotation
      \fi}
      {\if@twocolumn\else\endquotation\fi}
\fi
\makeatother

\begin{document}
\title{GIM - Team G Instant Messenger}
\author{Ewan Baird \\
Heather Hoaglund-Biron \\
Gordon Martin \\
James McMinn}
\date{21 March 2011}
\maketitle

\begin{abstract}
GIM is a client/server instant messaging application implemented in the Java Language. Comparable to many other similar programs, GIM allows users to communicate over a network with text based messages, using our own proprietary protocol. This dissertation discusses the process of developing GIM from the initial problem definition, though to completion.
\end{abstract}

\begin{acknowledgements}
Thanks to Colin Perkins for supervising this project, and being the voice of reason in those rare moments when panic set in.

For his excellent art work, we would like to thank our fellow student Matt Roszak. Without any brief, he provided an excellent logo and icon set for us, giving up his spare time in the process.

We would also like to thanks all of those who helped test the program, even in those early stages without the pretty GUI.
\end{acknowledgements}

\educationalconsent

\tableofcontents

\chapter{Introduction}
\label{introduction}

\section{Introduction}

Instant messengers are an easy, fast way to communicate over the Internet. Basic instant messengers allow users to type messages to each other on separate computers and have those messages immediately show on the screen, like instant email. More advanced systems have features like audio and video communication. These days instant messengers are becoming more popular, given the way the Internet is helping people from all over the world connect with one another. This has led to an increased demand for quick and easy communication.

These days, it's difficult to be original when developing an instant messenger. Programs like Skype, AIM and MSN Messenger are popular, especially with the younger generation. There are many kinds of instant messengers out there, the most prominent of which are always competing against each other to have the newest and most intriguing features.

This project is not about joining that competition. Instead, we are using this opportunity to explore what it's like to create an instant messenger from the ground up. As we are part of the younger generation, we use instant messengers almost daily, and it's interesting to be able to go behind the scenes and discover how they work for ourselves.

The instant messenger model that we decided to use consists of a server, any number of clients, and a set of rules defining how the server and clients talk to each other, called a protocol. A user of the program is essentially interacting with a client, and the clients interact with each other through the server. That communication is structured using the protocol.

Instead of using the protocol and server of an existing instant messenger and focusing on the client, we decided to create the entire system ourselves. This way we're able to learn more about the whole process of creating such a program. We have divided ourselves into two smaller groups: two of us to do the networking and make the server, and the other two to make the client, including the user interface, or UI.

This report is meant for the Computing Science professors at Glasgow University, fellow Computing Science students, and anyone with an interest in instant messaging or the process of completing a large-scale Computing Science project.

The structure of this report is as follows. In Chapter 1, we have this introduction, the problem definition, and an overview of the requirements. Chapter 2 covers the design of the protocol, client, server, and overall networking. In Chapter 3 is a discussion of our implementation of the client and server. Chapter 4 covers evaluation of our finished product, and Chapter 5 is a conclusion and reflective of what we learned.


\section{Problem Definition}
Instant messaging systems such as Jabber, AIM, ICQ and IRC have become popular in the last few years. The aim of the project is to build a simple instant messaging system, written in Java, to develop your understanding of networked systems and programming.

The group will be required to develop both an instant messaging client and server, and to design the network protocol they use to communicate. The server should accept connections from an arbitrary number of clients. Clients will have a graphical user interface, and should be able to accept and receive text-based messages, to indicate if the user is busy, available or idle, and to convey usernames and other details.

The project involves network and user-interface programming. It would suit a group with an interest in low-level systems design and implementation issues.



\chapter{Requirements}
\label{requirements}

This chapter contains the problem our team was given and how we interpreted it to find the required features of the system. After that, each feature we aimed to include will be listed in order from most to least important and explained in detail.

\section{Problem Definition}
Instant messaging systems such as Jabber, AIM, ICQ and IRC have become popular in the last few years. The aim of the project is to build a simple instant messaging system, written in Java, to develop your understanding of networked systems and programming.

The group will be required to develop both an instant messaging client and server, and to design the network protocol they use to communicate. The server should accept connections from an arbitrary number of clients. Clients will have a graphical user interface, and should be able to accept and receive text-based messages, to indicate if the user is busy, available or idle, and to convey usernames and other details.

The project involves network and user-interface programming. It would suit a group with an interest in low-level systems design and implementation issues.


\section{Features}

Once we established how to structure user communication, the next task was to determine what the important features of an instant messenger were, and what was achievable within the scope of the project. This process involved several team meetings where we discussed our experience with a variety of existing instant messengers and picked areas that we wished to draw from.

The many messaging programs already out there provided us with a solid foundation for our own client GUI. Our experiences with these programs allowed us to choose features which we felt were achievable and, more importantly, useful to users. We conceived a feature set split into 4 categories of importance, using the MoSCoW\footnote{\texttt{http://www.coleyconsulting.co.uk/moscow.htm}} method.

\subsection*{Must Have Features}

\begin{itemize}
\item{Send Messages\\
	The ability to send messages from one client to another.}
\item{Graphical User Interface (GUI)\\
	A collection of panels and buttons to help the user interact with the client.}
\item{User Nickname\\
	A name that the user can edit, used so that their contacts may more easily identify them.}
\item{Contact List\\
	A list of the user's contacts, viewable by the user.}
\item{User Status\\
	An attribute that shows whether the user is available to receive messages, which the user can set.}
\end{itemize}

The `Must Have' category has features that were taken from the initial problem specification, containing the basic elements of an instant messenger. Sending messages, using a GUI, and user statuses were taken from the problem definition, however we felt it was crucial and within the scope of this project to include contact lists and user nicknames as must have features.


\subsection*{Should Have Features}

\begin{itemize}
\item{URL Parsing\\
	The ability for a user to quickly select hyperlinks in the chat window.}
\item{Display Picture\\
	An image that a user uses to represent themselves to their contacts.}
\item{File Transferring\\
	The ability to send files from one client to another.}
\item{Personal Message\\
	A small message that the user can use to share news or anything else interesting.}
\item{Smileys\\
	Also known as emoticons, these are small icons the user can type, used to represent emotions in chat.}
\end{itemize}

`Should Have' features are those which we felt were within the scope of the project and would significantly enhance the user's experience. URL parsing was given high priority due to our experiences using other IM clients, which often involves sending various website links to contacts. While display pictures do not directly impact the functionality of the program, we felt that they would make the chatting experience more personal, and allow users to have what has been a standard feature of similar programs for some time. We considered the ability to send files between users a useful feature to include, despite the fact it would potentially be one of the most difficult items on the list to implement. As we considered personal messages simple to implement, it was assigned a relatively high priority. And while smileys add visual appeal, they would not add significant functionality since text-based representations can be used. They are easy enough to implement though, so we chose to have them in the second-highest category.


\subsection*{Could Have Features}

\begin{itemize}
\item{Contact List Grouping\\
	This is where the user can create subsets of their contacts. Managing large contact lists thus becomes much easier.}
\item{Offline Messaging\\
	This allows users to send messages to each other regardless of their status. When a user logs into the system, they receive any messages that were sent to them while they were not logged in.}
\item{Chat Logging\\
	The ability to store previous conversations on the user's computer.}
\item{Custom Fonts and Colours\\
	These options let users type in font colors and styles of their choosing.}
\end{itemize}

`Could Have' features are those that, given time, we think would be worth implementing. Contact list grouping would initially be in the form of displaying `Online' and `Offline' contacts. Time permitting, we might extend it to include user-defined custom groups. Offline messaging is useful for short disconnects to prevent messages being lost, so we thought it important enough to include. Chat logging can be useful, but from our experience, it is not used that often, and thus was not higher in the list of features. Custom fonts and colors can help make reading messages more pleasing to the eye, but aren't that important or required for functionality, thus the lower priority.


\subsection*{Would Like To Have Features}

\begin{itemize}
\item{User Profile\\
	User profiles are pages which contain details on a particular user, viewable by contacts.}
\item{Custom Commands\\
	These act in a similar way to an IRC client: when typed into the chat box, `/whois' and `/join', for example, show information about a user or join a chatroom, respectively. As these are `custom' commands, the user can change or add commands themselves.}
\item{Themes\\
	A preset of colours used to alter the interface's look and feel.}
\item{Plug-in Support\\
	The ability for other developers to create additional features that can be added to certain areas, for example support for other messenger protocols.}
\item{Voice-over Internet Protocol (VoIP)\\
	A protocol designed to accommodate voice chat between clients. A notable open standard of this is SIP, Session Initiation Protocol (RFC 3261\footnote{\texttt{http://tools.ietf.org/html/rfc3261}}).}
\end{itemize}

Features which we `Would Like To Have' are the lowest priority, and thus aren't important to implement. Beyond the basic concept of profiles, we had not decided how these would function or what they would contain. We did not think they were importantto include, as they aren't necessary for communication, and many instant messengers don't use them. Custom commands seemed unnecessary, since they require an overhead of knowing what the commands are and how they work, and we can provide functionality well enough with a GUI. We decided themes didn't seem part of useful functionality, but might be nice to add if we had time. Two of the more demanding requirements we conceived were plug-in support and VoIP. While these were desirable features, they were considered to be out of the scope of the project. Despite that fact, we decided to include them so we could aim to implement them at a later time.


\subsection*{Rejected Features}

There were some features which we agreed were not desirable or worth our time implementing.

\begin{itemize}

\item{Nudges\\
	Events which one user sends to another to force their chat window into focus.}
\item{Winks\\
	A similar feature to nudges, but containing multi-media.}
\item{Spell Checking\\
	The ability for a chat window text box to highlight misspelled words that the user has typed.}

\end{itemize}

Nudges and winks are considered by many to be an irritation which is prone to abuse, hence our decision to not include them. Spell Checking was considered, but due to the typically informal nature of IM conversations and the fact that people purposely misspell words at times, it was not included.




\chapter{Design}
\label{design}

The following chapter explains how we designed the system. As it is the most crucial part of the project, the protocol design will be explained first. Without a protocol that worked well, our system would not have been functional. Then the client and server design will be explained in detail, discussing how we've chosen to model the system. The last section will discuss in more detail how the networking between client and server was designed.

\newcommand\SLASH{\char`\\}

\section{Protocol Design}

\subsection{Protocol Overview}

\begin{figure}[!h]
    \begin{center}
        \includegraphics[scale=0.65]{chapter2/diagrams/protocol_high_level.png}
        \caption{The exchange of messages between a client and the server}
        \label{highLevelDia}
    \end{center}
\end{figure}

Gim uses a client-server architecture, where one computer (known as the Sever) acts as a central point to which other computers (the Clients) connect. The clients do not communicate directly with each other and all communication takes place between the clients and the server. If a client wishes to send a message to another client it must first got through the server.

In GIM, the Protocol is responsible for enabling communication between the clients and the server in a reliable and consistent manner. A protocol is a set of rules that determine the format and transmission of data between computers. The syntax (the structure, or format) and semantics (the meaning) of the Protocol are discussed in the next section.

At the highest level of abstraction, the GIM Protocol works in a very simple manner.  A client connects to the sever and they exchange messages until the connection is closed, as shown in Figure ~\ref{highLevelDia}.

In practice there are several clients connected to the server at once, however as the clients do not directly interact with each other and do not need to know about each other, the entire system can be described as above.

\subsection{Protocol Specification}
\subsubsection{Syntax}

The GIM Protocol uses a simple, text based syntax. Each full command begins with a colon followed by a command name and its arguments. A second colon marks the end of the headers and beginning of the data segment. A semi-colon marks the end of the data segment and the termination of the command. The basic structure of a full command is shown below:

\texttt{:<COMMAND\_NAME> <ARGUMENTS>: <DATA>;}

Commands names are predefined and \texttt{<COMMAND\_NAME>} can be any 1 of the following 19 values:

\texttt{
\begin{tabular}{ | p{3.3cm} |p{3.3cm}  |p{3.3cm} |p{3.3cm} | } 
    \hline
    AUTH & FRIENDREQUEST & MESSAGE & ROOM \\
    \hline
    BROADCAST & GET & OKAY & SERVERSTATUS \\
    \hline
    ERROR & INFO &PING & SET \\
    \hline
    FRIEND & KILL & PONG & UPDATE \\
    \hline
    FRIENDLIST & LOGOUT & QUIT & \\
    \hline
 \end{tabular}
 }
 
\texttt{<ARGUMENTS>} is a set of 0 or more predefined arguments which can be passed to the command in order to change its behaviour. Each command has its own set of arguments, some commands require arguments, some commands accept more than one argument and some commands have no arguments. The protocol specification defines over 60 arguments and a full listing for each command can be found in the \emph{GIM Protocol Specification Document}.

Unlike the command and arguments segments, the \texttt{<DATA>} segment does not have predefined values and has a varying syntax across different commands. The majority of the data is likely to have been provided by the user at some point, and as such the data segment is considered to be ``unsafe" and is encoded (See \emph{Encoding, Limits and Restrictions} for information about how the data is encoded). A full specification for the data segment of each command can be found in the \emph{GIM Protocol Specification Document}.

For example, the following is an example of a command sent from the client to the server requesting the Nickname and Status of the users joe@example.com and bill@gmail.org:

\texttt{:GET NICKNAME STATUS: joe\SLASH U+0040example.com bill\SLASH U+0040mail.org;}

To which the server may reply:

\texttt{:INFO NICKNAME STATUS: joe\SLASH U+0040example.com\\
Jeo\\
ONLINE\\
bill\SLASH U+0040mail.org\\
Billy The Kid\\
AWAY;}

\emph{(Please note that the data segments is the previous examples have been encoded as is defined below in Encoding, Limits and Restrictions.)}

\subsubsection{Roles and States}
The protocol defines two separate roles: The Client role and the Server role. Each role is only able to send a specific subset of commands but must also be able to receive and understand all of the commands sent by the opposite role. This means that both the client and server must implement all of the commands in some way.

Having clearly defined roles is an integral part of the GIM Protocol. This allows the semantics of a command to be used to generate a response in cases where both roles are able to send and receive a command but where the command has a different meaning depending on its source. 

For example, the \texttt{AUTH} command can be sent by both the client and the server roles. When sent by the server it is used to indicate the current state of the client (Either Logged-in or Unauthorized) but when sent by the client it indicates that they wish to login or register a new account. This means that the same command must be implemented differently depending on the role of the sender.

As mentioned above, the client role has 2 states: Logged-in and Unauthorized. When the client first establishes a connection it is placed in the unauthorized state. This means that it has access to an even smaller subset of commands, only those essential to logging-in or registering a new account. Once the client has successfully logged-in it is granted access to the full set of client commands.

\subsubsection{Security and Encryption}

\subsubsection{Encoding, Limits and Restrictions}

\subsection{Protocol Evolution}


\section{Client Evaluation}
\label{client_eval}




\section{Server Design}
\label{ServerDesign}

The GIM server is primarily responsible for synchronising the communication between the connected clients and keeping a persistent record of user information. In particular the server is responsible for:

\begin{itemize}
    \item{Authenticating users}
    \item{Routing chat messages for one client to one or more other clients}
    \item{Sending requests from one client to another client (such as friend requests or chat invites)}
    \item{Notifying subscribed users of changes to another user's information}
    \item{Storing user information such as nicknames, passwords, display pictures, friend lists, etc.}
\end{itemize}

This section discusses how the design of the server copes with these requirements and the the reasoning behind these decisions.

\subsection{Server Structure}

From the very beginning the server was designed to be secure, scalable, and simple. Users must trust it to keep their information safe and route messages correctly, and it must be able to cope with any number of connected clients.

In order to manage this, the server generates a new worker for every connected client, as shown in figure \ref{WorkersDia}. This new worker acts as the single point of contact for its respective client. This greatly simplifies the design of the server as we can treat each connection individually and allows the server to easily scale to a large number of clients.

\begin{figure}
    \begin{center}
        \includegraphics[scale=0.6]{Design/diagrams/server_workers.png}
        \caption{The GIM Server with 3 connected clients, each with their own worker.}
        \label{WorkersDia}
    \end{center}
\end{figure}

Each worker has two buffers, one which stores commands read from the network (the command buffer), and one which stores commands to be written to the network (the response buffer). The worker continually reads commands from the command buffer, processes them, and puts the responses into the response buffer. At the same time the worker is also performing network reads and writes to fill and empty the respective buffers. This is show in figure \ref{WorkerDatailedDia}.

The buffers allow the worker to run using several threads which increases the performance on multi-core systems as the server is able to read from the network, process commands, and write to the network in a truly concurrent fashion. This means that at any one time a worker could be handling up to three different commands, each at a different stage of execution, and still have several commands stored in the buffers waiting to be used.

\begin{figure}
    \begin{center}
        \includegraphics[scale=0.6]{Design/diagrams/worker_detail.png}
        \caption{The internal components of a single worker.}
        \label{WorkerDatailedDia}
    \end{center}
\end{figure}

\subsection{User Accounts}
A user is a registered account on the server with a unique identifier (an email address), password, nickname, personal message, display picture, status, and friend list. 

\begin{itemize}
\item{{\bf Identification}\\
The user's identifier and password are used to authenticate the user at login. These cannot be changed one the account has been created.} 

\item{{\bf Nickname}\\
A user's nickname is a short and familiar name for that user. A nickname does not need to be unique and is not intended to be used as an identifier by the system. By default a user's nickname is their email address.}

\item{{\bf Personal Message}\\
The personal message is a medium length message (usually only a sentence or two) which the user can use to share news or any information about themselves.}

\item{{\bf Status}\\
A user can have one of several statuses: Online, Away, Busy and Appear Offline. The first three statuses are used simply as an indication of the user's likelihood to respond to a message. The Appear Offline status allows the user to appear to other users as though they were not signed in but still receive updates and messages from other users.}

\item{{\bf State}\\
A user can have one of only two states, Online or Offline. States should not be confused with a user's status. A users status is simply an abstract representation, picked by the user, while a users state is an internal representation of the actual online state of the user. A user is only considered to be Online if they are connected using a client, and that client has logged-in successfully, otherwise they are considered to be Offline.}

\item{{\bf Friend List}\\
Each user has their own Friend List, a list of users who have accepted a friend request from the owner of the Friend List. When a user accepts a friend request, the sender and receiver are added to each other's Friend List.}
\end{itemize}



\subsection{Client-to-Client Messaging}
\label{c2c}
No direct client-to-client communication is possible. However, workers and (in an ad-hoc fashion) clients are able to indirectly communicate with other clients (i.e. clients to which they are not connected) by passing commands to the other client's worker. This is the server's main method of inter-client communication and relies heavily on the use of the buffers to pass messages from one worker to the network writer or command processor of another worker.

As the protocol defines no client to client communication is possible before a user has logged in, and as user can only be logged in on one client at a time, only a user id is necessary to send a command to a user. As long as a record is kept of which worker belongs to each online user, then sending a message to a user is no more complicated than identifying the particular user.

\subsection{Rooms and Message Routing}
\label{message_routing}

Message routing in the server is fairly simple because of the use of rooms. A room is a collection of one or more users and each room has a unique identifier. In order to join a room a user must either create a new room (in which case they are automatically placed into the room) or they must be invited to the room by a user already in the room. A room can be one of two types: personal or group. A personal room is much more restrictive than a group room. A personal room must be provided with the identifier of one other user at time of creation and is limited to only 2 users. Once a personal room has been created you cannot invite any other users. A group room has no such restrictions. Any number of users may be invited to a group room at any time.

All chat messages in GIM are sent to Rooms rather than particular users. The server must maintain a list of the users in each room, and upon receipt of a message for that room, forward it to all of the users in the room (excluding the sender of the message). This is done using the method described in section \ref{c2c}. A \texttt{:MESSAGE:} command is generated and placed into the Response Buffer of the worker assigned to each of the recipients ready for the worker to send to the client.

\subsection{Subscriptions, Privacy and Notifications}
One of the server's most critical responsibilities is notifying users of changes to other users. This is done using the concept of subscriptions. A user (the subscriber) is considered to be subscribed to another user (the subscribee) if they meet at least one of two conditions:

\begin{enumerate}
\item{The subscriber has the subscribee in their friend list, or had them in their friend list at some point in the past}
\item{Both the subscriber and subscribee are in the same room}
\end{enumerate}

When a user changes any piece of their user information such as their status, nickname, or personal message, the server needs to notify other clients so that they can update their information. The server generates a list of subscribers for this user and removes offline and blocked users. It then issues an \texttt{:UPDATE:} command to each of the clients, notifying them of the user and the piece of information which has been updated. Note that the server does not send the updated information, it merely notifies the client of the update, allowing it to request the updated information for the server when and if it needs it.

Subscriptions are also used as a method of access control. A user is only able to request information about another user if they are subscribed to that user. This limitation means that a user is effectively granting another user access to read their information by adding them to their friend list. However, the converse is not true. Removing a user from your friend list does not remove their access to your information as they may still have you in their friend list. In order to combat this, you are able to block a user which removes this access to your information and status updates. This enhances privacy and gives the user control of who is able to access their data.

\subsection{Synchronisation}
The server has a large section of backing data which stores all of the users, rooms, workers and any other operational data required by the server. Due to the highly concurrent nature of the server it is very important to ensure that the data is properly synchronised. This is to prevent race conditions where two writes occur on an object at the same time giving an unpredictable output, as show in \ref{RaceConditionDia}.

\begin{figure}
    \begin{center}
        \includegraphics[scale=0.6]{Design/diagrams/server_race_condition.png}
        \caption{Unsynchronised workers updating the same object at the same time, resulting in a race condition and giving the unexpected value of 6 rather than 7.}
        \label{RaceConditionDia}
    \end{center}
\end{figure}

To stop this the server must lock (claim ownership of) the data before it is updated. This is a very tricky thing to do as as the lock must be long enough to ensure that no updates occur between reading the original data and writing the updated data, while also keeping the lock as short as possible to avoid long periods of blocking and keep performance high.

The server must also make sure only to lock when needed, and not obtain a lock for objects it does not require. This simplest solution would be to lock the entire data object, and only allow reads and writes through a proxy. However this would be extremely costly in terms of performance as writes to individual objects inside the data object would block writes to other objects, meaning that at any one time only a single write could occur. Instead the data object must be clever and only lock the objects which are to be written to, allowing for simultaneous writes to separate objects while still stopping race conditions from occurring.

\begin{figure}
    \begin{center}
        \includegraphics[scale=0.6]{Design/diagrams/server_locking.png}
        \caption{Synchronised workers using locking to read and write the updated value, giving the expected value of 7.}
        \label{lockingDia}
    \end{center}
\end{figure}

\subsection{The Controller and Timeout Manager}
Even though the server should be capable of functioning autonomously, some of the features defined by the protocol require interaction with the server that deviates from the client-server model. For example, the \texttt{:BROADCAST:} command allows the server to send a message to all of the clients it is connected to. The controller allows authorised users to interact through a command line interface with the server. These authorised users are allowed by the operating system to connect to the session that is running the server. They are then able to execute several commands defined by the controller to broadcast messages and control the server. These commands are independent from protocol commands.

For example, the \texttt{:QUIT:} command tells the server to shut down and create a persistence file to store user data, and the following would send a \texttt{:BROADCAST:} command to all of the connected clients:

\texttt{BROADCAST We're shutting the server down in 15 minutes.}

The server needs to periodically check for clients which have not sent a command in the last 15 seconds (as defined in the protocol) and disconnect them. This means that the server must compare the time the last command was received by every worker with the current time. If the difference exceeds 15 seconds, they are disconnected.

Due to the nature of the Controller and Timeout Manager, they both must run on separate threads from the rest of the server or risk blocking a more important function, such as accepting incoming connections and creating workers for them.

\begin{landscape}
    \begin{figure}
        \begin{center}
            \includegraphics[scale=0.5]{Design/diagrams/server_uml.png}
            \caption{UML showing the structure of the GIM Server.}
            \label{umlDia}
        \end{center}
    \end{figure}
\end{landscape}


\section{Networking Design}

\subsection{Client Networking Structure}

We determined the role of the client networking component to be to maintain a connection with the server, read commands from the command buffer and pass them to the server, and pass commands from the server to the client. These two concerns came with different challenges; where keeping a connection with the server involved considering how to manage writing and reading data to the socket simultaneously, and sending and receiving commands involved considering how best to parse data coming from the socket, and pass data between components.

\subsubsection {Maintaining a Connection}

The connection to the server had to be designed such that our program could freely write to the socket, and receive commands with the appearance of concurrency. Furthermore, there client code should not have to worry about doing two simultaneous writes to the server, if there is any multithreaded aspects that could cause this. In reality, we would have to manage the connection to the socket so that the socket would not be written to while a read or write was in progress. 



\subsubsection {Command Handling }












%TODO: Chapter conclusion. Reflect on your design, and summarize the chapter. Explain how it relates to the implementation chapter.


\chapter{Implementation}
\label{implementation}

\section{Client Evaluation}
\label{client_eval}




\section{Server Design}
\label{ServerDesign}

The GIM server is primarily responsible for synchronising the communication between the connected clients and keeping a persistent record of user information. In particular the server is responsible for:

\begin{itemize}
    \item{Authenticating users}
    \item{Routing chat messages for one client to one or more other clients}
    \item{Sending requests from one client to another client (such as friend requests or chat invites)}
    \item{Notifying subscribed users of changes to another user's information}
    \item{Storing user information such as nicknames, passwords, display pictures, friend lists, etc.}
\end{itemize}

This section discusses how the design of the server copes with these requirements and the the reasoning behind these decisions.

\subsection{Server Structure}

From the very beginning the server was designed to be secure, scalable, and simple. Users must trust it to keep their information safe and route messages correctly, and it must be able to cope with any number of connected clients.

In order to manage this, the server generates a new worker for every connected client, as shown in figure \ref{WorkersDia}. This new worker acts as the single point of contact for its respective client. This greatly simplifies the design of the server as we can treat each connection individually and allows the server to easily scale to a large number of clients.

\begin{figure}
    \begin{center}
        \includegraphics[scale=0.6]{Design/diagrams/server_workers.png}
        \caption{The GIM Server with 3 connected clients, each with their own worker.}
        \label{WorkersDia}
    \end{center}
\end{figure}

Each worker has two buffers, one which stores commands read from the network (the command buffer), and one which stores commands to be written to the network (the response buffer). The worker continually reads commands from the command buffer, processes them, and puts the responses into the response buffer. At the same time the worker is also performing network reads and writes to fill and empty the respective buffers. This is show in figure \ref{WorkerDatailedDia}.

The buffers allow the worker to run using several threads which increases the performance on multi-core systems as the server is able to read from the network, process commands, and write to the network in a truly concurrent fashion. This means that at any one time a worker could be handling up to three different commands, each at a different stage of execution, and still have several commands stored in the buffers waiting to be used.

\begin{figure}
    \begin{center}
        \includegraphics[scale=0.6]{Design/diagrams/worker_detail.png}
        \caption{The internal components of a single worker.}
        \label{WorkerDatailedDia}
    \end{center}
\end{figure}

\subsection{User Accounts}
A user is a registered account on the server with a unique identifier (an email address), password, nickname, personal message, display picture, status, and friend list. 

\begin{itemize}
\item{{\bf Identification}\\
The user's identifier and password are used to authenticate the user at login. These cannot be changed one the account has been created.} 

\item{{\bf Nickname}\\
A user's nickname is a short and familiar name for that user. A nickname does not need to be unique and is not intended to be used as an identifier by the system. By default a user's nickname is their email address.}

\item{{\bf Personal Message}\\
The personal message is a medium length message (usually only a sentence or two) which the user can use to share news or any information about themselves.}

\item{{\bf Status}\\
A user can have one of several statuses: Online, Away, Busy and Appear Offline. The first three statuses are used simply as an indication of the user's likelihood to respond to a message. The Appear Offline status allows the user to appear to other users as though they were not signed in but still receive updates and messages from other users.}

\item{{\bf State}\\
A user can have one of only two states, Online or Offline. States should not be confused with a user's status. A users status is simply an abstract representation, picked by the user, while a users state is an internal representation of the actual online state of the user. A user is only considered to be Online if they are connected using a client, and that client has logged-in successfully, otherwise they are considered to be Offline.}

\item{{\bf Friend List}\\
Each user has their own Friend List, a list of users who have accepted a friend request from the owner of the Friend List. When a user accepts a friend request, the sender and receiver are added to each other's Friend List.}
\end{itemize}



\subsection{Client-to-Client Messaging}
\label{c2c}
No direct client-to-client communication is possible. However, workers and (in an ad-hoc fashion) clients are able to indirectly communicate with other clients (i.e. clients to which they are not connected) by passing commands to the other client's worker. This is the server's main method of inter-client communication and relies heavily on the use of the buffers to pass messages from one worker to the network writer or command processor of another worker.

As the protocol defines no client to client communication is possible before a user has logged in, and as user can only be logged in on one client at a time, only a user id is necessary to send a command to a user. As long as a record is kept of which worker belongs to each online user, then sending a message to a user is no more complicated than identifying the particular user.

\subsection{Rooms and Message Routing}
\label{message_routing}

Message routing in the server is fairly simple because of the use of rooms. A room is a collection of one or more users and each room has a unique identifier. In order to join a room a user must either create a new room (in which case they are automatically placed into the room) or they must be invited to the room by a user already in the room. A room can be one of two types: personal or group. A personal room is much more restrictive than a group room. A personal room must be provided with the identifier of one other user at time of creation and is limited to only 2 users. Once a personal room has been created you cannot invite any other users. A group room has no such restrictions. Any number of users may be invited to a group room at any time.

All chat messages in GIM are sent to Rooms rather than particular users. The server must maintain a list of the users in each room, and upon receipt of a message for that room, forward it to all of the users in the room (excluding the sender of the message). This is done using the method described in section \ref{c2c}. A \texttt{:MESSAGE:} command is generated and placed into the Response Buffer of the worker assigned to each of the recipients ready for the worker to send to the client.

\subsection{Subscriptions, Privacy and Notifications}
One of the server's most critical responsibilities is notifying users of changes to other users. This is done using the concept of subscriptions. A user (the subscriber) is considered to be subscribed to another user (the subscribee) if they meet at least one of two conditions:

\begin{enumerate}
\item{The subscriber has the subscribee in their friend list, or had them in their friend list at some point in the past}
\item{Both the subscriber and subscribee are in the same room}
\end{enumerate}

When a user changes any piece of their user information such as their status, nickname, or personal message, the server needs to notify other clients so that they can update their information. The server generates a list of subscribers for this user and removes offline and blocked users. It then issues an \texttt{:UPDATE:} command to each of the clients, notifying them of the user and the piece of information which has been updated. Note that the server does not send the updated information, it merely notifies the client of the update, allowing it to request the updated information for the server when and if it needs it.

Subscriptions are also used as a method of access control. A user is only able to request information about another user if they are subscribed to that user. This limitation means that a user is effectively granting another user access to read their information by adding them to their friend list. However, the converse is not true. Removing a user from your friend list does not remove their access to your information as they may still have you in their friend list. In order to combat this, you are able to block a user which removes this access to your information and status updates. This enhances privacy and gives the user control of who is able to access their data.

\subsection{Synchronisation}
The server has a large section of backing data which stores all of the users, rooms, workers and any other operational data required by the server. Due to the highly concurrent nature of the server it is very important to ensure that the data is properly synchronised. This is to prevent race conditions where two writes occur on an object at the same time giving an unpredictable output, as show in \ref{RaceConditionDia}.

\begin{figure}
    \begin{center}
        \includegraphics[scale=0.6]{Design/diagrams/server_race_condition.png}
        \caption{Unsynchronised workers updating the same object at the same time, resulting in a race condition and giving the unexpected value of 6 rather than 7.}
        \label{RaceConditionDia}
    \end{center}
\end{figure}

To stop this the server must lock (claim ownership of) the data before it is updated. This is a very tricky thing to do as as the lock must be long enough to ensure that no updates occur between reading the original data and writing the updated data, while also keeping the lock as short as possible to avoid long periods of blocking and keep performance high.

The server must also make sure only to lock when needed, and not obtain a lock for objects it does not require. This simplest solution would be to lock the entire data object, and only allow reads and writes through a proxy. However this would be extremely costly in terms of performance as writes to individual objects inside the data object would block writes to other objects, meaning that at any one time only a single write could occur. Instead the data object must be clever and only lock the objects which are to be written to, allowing for simultaneous writes to separate objects while still stopping race conditions from occurring.

\begin{figure}
    \begin{center}
        \includegraphics[scale=0.6]{Design/diagrams/server_locking.png}
        \caption{Synchronised workers using locking to read and write the updated value, giving the expected value of 7.}
        \label{lockingDia}
    \end{center}
\end{figure}

\subsection{The Controller and Timeout Manager}
Even though the server should be capable of functioning autonomously, some of the features defined by the protocol require interaction with the server that deviates from the client-server model. For example, the \texttt{:BROADCAST:} command allows the server to send a message to all of the clients it is connected to. The controller allows authorised users to interact through a command line interface with the server. These authorised users are allowed by the operating system to connect to the session that is running the server. They are then able to execute several commands defined by the controller to broadcast messages and control the server. These commands are independent from protocol commands.

For example, the \texttt{:QUIT:} command tells the server to shut down and create a persistence file to store user data, and the following would send a \texttt{:BROADCAST:} command to all of the connected clients:

\texttt{BROADCAST We're shutting the server down in 15 minutes.}

The server needs to periodically check for clients which have not sent a command in the last 15 seconds (as defined in the protocol) and disconnect them. This means that the server must compare the time the last command was received by every worker with the current time. If the difference exceeds 15 seconds, they are disconnected.

Due to the nature of the Controller and Timeout Manager, they both must run on separate threads from the rest of the server or risk blocking a more important function, such as accepting incoming connections and creating workers for them.

\begin{landscape}
    \begin{figure}
        \begin{center}
            \includegraphics[scale=0.5]{Design/diagrams/server_uml.png}
            \caption{UML showing the structure of the GIM Server.}
            \label{umlDia}
        \end{center}
    \end{figure}
\end{landscape}


\section{Client-Server Communication}
While no encryption is specifically defined in the protocol, the GIM client and server use SSL encryption to keep their communication secure. The \texttt{java.net.ssl} package provides access to all of classed needed to enable the secure connection, in particular the SSLSocket and SSLServerSocket classes.

When the server starts it creates an SSLServerSocket on port 4444. It then generates a list of supported cipher suites and filters them to only include ones which use 128bit AES encryption, limits the socket to using only these ciphers, and waits for incoming connections using the accept() method.

The client does something very similar. When it first starts it creates an SSLSocket on port 4444 using the hostname of the server (rooster.dyndns.info was used as the development server). It then generates a list of 128bit AES encryption ciphers, limits the socket to these and allows Java to negotiate a connection with the server using a cipher which is supported by both parties.

Once the server has accepted the connection and created a new worker for the client, both the client are server create new BufferedReaders and PrintWriters using the input and output streams of the socket. The sockets can then be used  with  the standard I/O methods built into Java, making communication over the socket no more complicated than writing to the console.


\section{Storage}
\label{storage}

Both the GIM client and server use Java's built-in serialization features to store data locally while they are not running. The Data class of the server and the Options class of the client both implement Serializable from \texttt{java.io}. This allows them to be serialized and stored on disk using an ObjectOutputStream object (also in \texttt{java.io}).

When the client starts it reads this file from the disk and uses an ObjectInputStream to turn the raw data back into the Options object. For the server, it is not quite so simple. Although the server serializes the entire Data object, some of the data stored in it (such as as rooms) does not need to be persisted across sessions and we create a section Data object and copy the persistent data from the serialized object into it.



\chapter{Evaluation}
\label{evaluation}

\section{Server Design}
\label{ServerDesign}

The GIM server is primarily responsible for synchronising the communication between the connected clients and keeping a persistent record of user information. In particular the server is responsible for:

\begin{itemize}
    \item{Authenticating users}
    \item{Routing chat messages for one client to one or more other clients}
    \item{Sending requests from one client to another client (such as friend requests or chat invites)}
    \item{Notifying subscribed users of changes to another user's information}
    \item{Storing user information such as nicknames, passwords, display pictures, friend lists, etc.}
\end{itemize}

This section discusses how the design of the server copes with these requirements and the the reasoning behind these decisions.

\subsection{Server Structure}

From the very beginning the server was designed to be secure, scalable, and simple. Users must trust it to keep their information safe and route messages correctly, and it must be able to cope with any number of connected clients.

In order to manage this, the server generates a new worker for every connected client, as shown in figure \ref{WorkersDia}. This new worker acts as the single point of contact for its respective client. This greatly simplifies the design of the server as we can treat each connection individually and allows the server to easily scale to a large number of clients.

\begin{figure}
    \begin{center}
        \includegraphics[scale=0.6]{Design/diagrams/server_workers.png}
        \caption{The GIM Server with 3 connected clients, each with their own worker.}
        \label{WorkersDia}
    \end{center}
\end{figure}

Each worker has two buffers, one which stores commands read from the network (the command buffer), and one which stores commands to be written to the network (the response buffer). The worker continually reads commands from the command buffer, processes them, and puts the responses into the response buffer. At the same time the worker is also performing network reads and writes to fill and empty the respective buffers. This is show in figure \ref{WorkerDatailedDia}.

The buffers allow the worker to run using several threads which increases the performance on multi-core systems as the server is able to read from the network, process commands, and write to the network in a truly concurrent fashion. This means that at any one time a worker could be handling up to three different commands, each at a different stage of execution, and still have several commands stored in the buffers waiting to be used.

\begin{figure}
    \begin{center}
        \includegraphics[scale=0.6]{Design/diagrams/worker_detail.png}
        \caption{The internal components of a single worker.}
        \label{WorkerDatailedDia}
    \end{center}
\end{figure}

\subsection{User Accounts}
A user is a registered account on the server with a unique identifier (an email address), password, nickname, personal message, display picture, status, and friend list. 

\begin{itemize}
\item{{\bf Identification}\\
The user's identifier and password are used to authenticate the user at login. These cannot be changed one the account has been created.} 

\item{{\bf Nickname}\\
A user's nickname is a short and familiar name for that user. A nickname does not need to be unique and is not intended to be used as an identifier by the system. By default a user's nickname is their email address.}

\item{{\bf Personal Message}\\
The personal message is a medium length message (usually only a sentence or two) which the user can use to share news or any information about themselves.}

\item{{\bf Status}\\
A user can have one of several statuses: Online, Away, Busy and Appear Offline. The first three statuses are used simply as an indication of the user's likelihood to respond to a message. The Appear Offline status allows the user to appear to other users as though they were not signed in but still receive updates and messages from other users.}

\item{{\bf State}\\
A user can have one of only two states, Online or Offline. States should not be confused with a user's status. A users status is simply an abstract representation, picked by the user, while a users state is an internal representation of the actual online state of the user. A user is only considered to be Online if they are connected using a client, and that client has logged-in successfully, otherwise they are considered to be Offline.}

\item{{\bf Friend List}\\
Each user has their own Friend List, a list of users who have accepted a friend request from the owner of the Friend List. When a user accepts a friend request, the sender and receiver are added to each other's Friend List.}
\end{itemize}



\subsection{Client-to-Client Messaging}
\label{c2c}
No direct client-to-client communication is possible. However, workers and (in an ad-hoc fashion) clients are able to indirectly communicate with other clients (i.e. clients to which they are not connected) by passing commands to the other client's worker. This is the server's main method of inter-client communication and relies heavily on the use of the buffers to pass messages from one worker to the network writer or command processor of another worker.

As the protocol defines no client to client communication is possible before a user has logged in, and as user can only be logged in on one client at a time, only a user id is necessary to send a command to a user. As long as a record is kept of which worker belongs to each online user, then sending a message to a user is no more complicated than identifying the particular user.

\subsection{Rooms and Message Routing}
\label{message_routing}

Message routing in the server is fairly simple because of the use of rooms. A room is a collection of one or more users and each room has a unique identifier. In order to join a room a user must either create a new room (in which case they are automatically placed into the room) or they must be invited to the room by a user already in the room. A room can be one of two types: personal or group. A personal room is much more restrictive than a group room. A personal room must be provided with the identifier of one other user at time of creation and is limited to only 2 users. Once a personal room has been created you cannot invite any other users. A group room has no such restrictions. Any number of users may be invited to a group room at any time.

All chat messages in GIM are sent to Rooms rather than particular users. The server must maintain a list of the users in each room, and upon receipt of a message for that room, forward it to all of the users in the room (excluding the sender of the message). This is done using the method described in section \ref{c2c}. A \texttt{:MESSAGE:} command is generated and placed into the Response Buffer of the worker assigned to each of the recipients ready for the worker to send to the client.

\subsection{Subscriptions, Privacy and Notifications}
One of the server's most critical responsibilities is notifying users of changes to other users. This is done using the concept of subscriptions. A user (the subscriber) is considered to be subscribed to another user (the subscribee) if they meet at least one of two conditions:

\begin{enumerate}
\item{The subscriber has the subscribee in their friend list, or had them in their friend list at some point in the past}
\item{Both the subscriber and subscribee are in the same room}
\end{enumerate}

When a user changes any piece of their user information such as their status, nickname, or personal message, the server needs to notify other clients so that they can update their information. The server generates a list of subscribers for this user and removes offline and blocked users. It then issues an \texttt{:UPDATE:} command to each of the clients, notifying them of the user and the piece of information which has been updated. Note that the server does not send the updated information, it merely notifies the client of the update, allowing it to request the updated information for the server when and if it needs it.

Subscriptions are also used as a method of access control. A user is only able to request information about another user if they are subscribed to that user. This limitation means that a user is effectively granting another user access to read their information by adding them to their friend list. However, the converse is not true. Removing a user from your friend list does not remove their access to your information as they may still have you in their friend list. In order to combat this, you are able to block a user which removes this access to your information and status updates. This enhances privacy and gives the user control of who is able to access their data.

\subsection{Synchronisation}
The server has a large section of backing data which stores all of the users, rooms, workers and any other operational data required by the server. Due to the highly concurrent nature of the server it is very important to ensure that the data is properly synchronised. This is to prevent race conditions where two writes occur on an object at the same time giving an unpredictable output, as show in \ref{RaceConditionDia}.

\begin{figure}
    \begin{center}
        \includegraphics[scale=0.6]{Design/diagrams/server_race_condition.png}
        \caption{Unsynchronised workers updating the same object at the same time, resulting in a race condition and giving the unexpected value of 6 rather than 7.}
        \label{RaceConditionDia}
    \end{center}
\end{figure}

To stop this the server must lock (claim ownership of) the data before it is updated. This is a very tricky thing to do as as the lock must be long enough to ensure that no updates occur between reading the original data and writing the updated data, while also keeping the lock as short as possible to avoid long periods of blocking and keep performance high.

The server must also make sure only to lock when needed, and not obtain a lock for objects it does not require. This simplest solution would be to lock the entire data object, and only allow reads and writes through a proxy. However this would be extremely costly in terms of performance as writes to individual objects inside the data object would block writes to other objects, meaning that at any one time only a single write could occur. Instead the data object must be clever and only lock the objects which are to be written to, allowing for simultaneous writes to separate objects while still stopping race conditions from occurring.

\begin{figure}
    \begin{center}
        \includegraphics[scale=0.6]{Design/diagrams/server_locking.png}
        \caption{Synchronised workers using locking to read and write the updated value, giving the expected value of 7.}
        \label{lockingDia}
    \end{center}
\end{figure}

\subsection{The Controller and Timeout Manager}
Even though the server should be capable of functioning autonomously, some of the features defined by the protocol require interaction with the server that deviates from the client-server model. For example, the \texttt{:BROADCAST:} command allows the server to send a message to all of the clients it is connected to. The controller allows authorised users to interact through a command line interface with the server. These authorised users are allowed by the operating system to connect to the session that is running the server. They are then able to execute several commands defined by the controller to broadcast messages and control the server. These commands are independent from protocol commands.

For example, the \texttt{:QUIT:} command tells the server to shut down and create a persistence file to store user data, and the following would send a \texttt{:BROADCAST:} command to all of the connected clients:

\texttt{BROADCAST We're shutting the server down in 15 minutes.}

The server needs to periodically check for clients which have not sent a command in the last 15 seconds (as defined in the protocol) and disconnect them. This means that the server must compare the time the last command was received by every worker with the current time. If the difference exceeds 15 seconds, they are disconnected.

Due to the nature of the Controller and Timeout Manager, they both must run on separate threads from the rest of the server or risk blocking a more important function, such as accepting incoming connections and creating workers for them.

\begin{landscape}
    \begin{figure}
        \begin{center}
            \includegraphics[scale=0.5]{Design/diagrams/server_uml.png}
            \caption{UML showing the structure of the GIM Server.}
            \label{umlDia}
        \end{center}
    \end{figure}
\end{landscape}


\section{Client Evaluation}
\label{client_eval}




\section{User Evaluation}
\label{user_eval}
blah blah blah



\chapter{Conclusion}
\label{conclusion}

Over the course of this project, we've all learned a lot as a team. The importance of communication and organisation, the design decisions required of a large project, and how to split up work fairly. We've also learned more about networking and making GUIs in Java Swing. The following is a more in-depth discussion of how the project went for us.

Early on, we had a sort of `laissez-faire' style of team management, with no one in particular in charge. We had few strict roles among us either, though a few had been decided. Gordon was willing to be secretary and take meeting minutes, and Heather was experienced with writing and editing, as well as being good at organisation, so she took the role of Quality Assurance. While we were designing the project though, we didn't have any other defined roles.

As we progressed further into the design, it was clear we needed to split up the work in a manageable way if we wanted things to be fair. We decided to have Gordon and James work on the server and Ewan and Heather to work on the client and corresponding GUI. This worked well, since James and Gordon had more of an interest in networking, and Ewan and Heather were more experienced with making GUIs. Soon we found we hadn't made a clear line between the server and client work, and thus some of the networking code was split between the two smaller teams. The client team would have had a lot more work than the server team, considering how much networking code it involved. We then came up with the idea of the interfaces\footnote{see gordon's design section}, which helped more precisely define the work boundary between the two teams. The server team implemented methods that the client would call, and the client team implemented methods that the server would call. This more fairly split up the work and made it manageable.

Over the first term, we met weekly as a team, in addition to our meetings with our supervisor. These supervisor meetings in particular helped us stay on track, since we went over what we did each week and then discussed our plans for the following weeks. The meetings also gave us formative feedback about our approach to the project and the dissertation. We likely would have been more disorganised without them. The team meetings were helpful too, as we all focused at the same time, in person, on what had to be done that week.

Then when the long winter break came, we decided to set up an IRC channel for ourselves to help communicate, especially since both Heather and James traveled and weren't able to meet with the rest of the team in person. This IRC channel helped us immensely, and we all had enough interest in the project to get a fair amount of coding done over the break. Once the next term had started, we had the client and server communicating mostly successfully. Over this term, we didn't have the same convenient hour between lectures to meet, and started to rely mostly on IRC for communication. In retrospect we probably would have benefited from continuing to meet every week, since at this point not all the team members were aware of what aspects of the project needed to be done. The use of tools like SVN\footnote{unless mentioned earlier}, though, were helpful in that regard, as we were able to read the commit logs and see what other members had recently done. Most of the work had been finished over the break anyway, and at this point there were mostly just little bugs to be fixed.

Laissez-faire management models don't always get good results, since some types of people need more direction, someone telling them what to do. Over the course of the year, we observed other teams struggling as some of their team members did little or no work towards their projects, even teams without the laissez-faire model. But we were lucky in that regard, as we were a team of individuals that were all ready and willing to learn and work with our team.

We've all learned a lot about dealing with large projects, as we have all never worked on a project of this size. Aside from team management, the amount of designing and coding in general was daunting at times. The fact that we were able to split up the work into teams of people who were good at what they were working on was very beneficial. In addition, coordinating how each component communicated and worked with one another--the GUI to the client model, for example--was a challenging part of this project. To overcome this, we spent time working out the confusion with each other, and ended up with a design that worked decently for the scope of this project.

Splitting up the work into teams had its downsides, though. As James and Gordon were focusing on the server and how it communicated with the client, they didn't get as much experience with GUIs and the HCI aspects of the project. Also, vice versa, Ewan and Heather didn't get as intimate knowledge of the workings of the server and its networking aspects as they would if they had been given tasks working with the server. We did have to learn what each other was working on in order to communicate, but we didn't learn as much about each other's work as we might have if we'd all had a hand in each part of the project. Despite this lack of learning, overall it did help the project progress faster since everyone was working on what they knew about.

Overall, we think this project has been a success. Maybe each of us separately hasn't learned every detail about how to create an instant messenger from scratch, but collectively we certainly have, and we've learned a lot about team and large project management in the process. Our team worked well together and communicated effectively enough, despite our slow disorganisation at the end. We hope that GIM will be used in some way in the future, perhaps studied by others interested in this topic, extended by one of us in a future project, or even used to do what it was made for: instant messaging.




\appendix
\chapter{GIM Protocol Specification}
\section{Introduction}

A network protocol is a set of rules which must be followed for a program to produce the desired effect on another machine in a network based system. The motivation behind creating a protocol is to allow a standard pattern of communication to be devised for a given system. This allows a system following the protocol to consistently perform the same role, in any context, regardless of its implementation. Any command in our protocol should produce a verifiable result as outlined in this section.

To guarantee any desired outcome, a ‘message’ must be well formed (follow the correct syntax), be sent while the machine is in a valid ‘state’, and be sent within any expected time frame. This document outlines these aspects of our protocol in detail - specifying any constraint on when an action may be undertaken, and any constraint on the time allowed for this action to be performed. The outcome on the system state will also be detailed. Both aspects can be identified by the pre-conditions and post-conditions surrounding any given command.

In our framework, there is one machine that undertakes a ‘server’ role, and many machines that can undertake a ‘client’ role. These roles govern what a machine may validly send or receive, depending on its role. A general a pattern of communication, such that one role sending a ‘message’ to the other results in a reply from this role, and a possible change of ‘state’ in each member, will be evidenced in this section. To better understand our protocol, it is  necessary to outline the function of these roles, with some reference to the states they maintain, to illuminate the utility of available messages.

At a high level, the role of the server is to keep a record of state persistent aspects of the user’s account; composed of details such as login details, a buddy list, and privacy conditions. Further, it holds volatile state information, relevant only to a session (which can be defined as a state where the user is logged in.) It monitors the global state of the system, an relays information such as which buddies are available to message, and their state. Its pivotal role is to relay messages between clients. 

The role of the client is to allow the user to retrieve, via an interface, information regarding their buddies, and to send messages to the server which will subsequently be relayed to their buddies. Furthermore, it allows the user to modify their state within the system via the server.

\section{Commands}

\subsection{Command Structure}

Each message will follow this high level syntax:
\texttt{:COMMAND [ ARG1 | ARG2 | ARG3 ]: <data>;}
- Square brackets [ ... ] denote a list of 1 or more arguments of which only one can be given and is required.  
- Parenthesis \{ ... \} are used to represent arguments which are optional and can be non-singular.
- A colon is used in the first instance to indicate the opening of a command, and in the second to indicate a boundary between a command and arguments, and the data set.
- Angular brackets < … > are used to show data that has been provided by the user and should be considered unsafe.
- A semi colon ; is used to show the end of a command.

By example:
\texttt{:GET \{ NICKNAME | STATUS | PERSONAL_MESSAGE | DISPLAY_PIC \}: <user> <user>...;}\\
Could be used as follows:
\textttt{:GET NICKNAME STATUS PERSONAL_MESSAGE: cyblob@idgmail.com meow@hotmail.com;}
Which means request the Nickname, Status and Personal Message of users cyblob@gmail.com and meow@hotmail.com.

Remarkably, this minimalistic structure is suitable to fulfil every requirement of the system, and promotes readability.

Every command or request which the sever receives must be followed by a response of some type. Where there is a command which does not return any data the server should response with the :OKAY: command to signal that the command was received and error free.

Commands have a maximum length of 8192 characters, including any punctuation, with the exception of the :SET DISPLAY_PIC: command, which has a character limit of 32768.

\subsection{Heart Beat}

In order for the server to recognise that the client is still functioning, it must receive a command from the client on a regular basis (around every 15 seconds). In the event that no command is received for an extended period, the server will disconnect the client.

Any command will reset the timeout, however it is reccomended that the PING command is used.

\section{Pre-login Commands}

These are the only commands possible before a user has logged in and remain possible after the user has logged in.

These include all commands necessary for establishing a connection, logging in to the server and creating a new account.

\subsection{Server Commands}

\texttt{:PING:;}

The command is used as a keep-alive command. Its primary use it to indicate that the client is still alive even if no other communication has been received from the client. 

Examples:
\texttt{:PING: ;} \\
The client responding to a :PING: commands

{\bf :SERVERSTATUS \{ USERS | TIME | UPTIME \}:;}

The SERVERSTATUS command returns information about the server including the number of users, the local system time and server up-time.
 
In that case that an argument is provided, the server should then return a SERVERSTATUS as defined in the client section. If more than one argument is provided, then the server should return each value on a new line in the order which the arguments were received (reading from left to right).

The arguments for this command are as follows:

\begin{itemize}

\item \texttt{USERS \\
The total number of users and number of online users}

\item \texttt{TIME \\
The current local time of the server}

\item \texttt{UPTIME \\
The up-time of the server instance}

\end{itemize}

Examples:
\texttt{:SERVERSTATUS USERS:;} \\
Response>:\texttt{SERVERSTATUS USERS: 23 Online, 565 Total}
\texttt{:SERVERSTATUS TIME:;} \\
Response>:\texttt{SERVERSTATUS TIME: Wed 17 Nov, 23:31}

{\bf :AUTH \{ LOGIN | REGISTER \}: <email address> <password>;}

The AUTH commands deals with all aspects of user authorisation and permissions. The AUTH command alone should should return the users' current authorisation state, either LOGGEDIN or UNAUTHORIZED.

{\bf LOGIN}

If the details are valid then the server should respond with an \texttt{:AUTH LOGGEDIN:} command. It should set the user’s state (but not status) to ONLINE.
In the event that an error occurs the server should generate one of the following ERROR statuses:

\texttt{USER_DOES_NOT_EXIST}\\
The email address has not been registered
\texttt{LOGIN_DETAILS_INCORRECT} \\
The password was incorrect
\texttt{MISSING_ARGUMENTS} \\
There were too few arguments

If the user is already logged in then the server should send a \texttt{LOGGED_IN_FROM_OTHER_LOCATION} error and KILL command to the already connected user.

{\bf REGISTER}

The register argument allows for new users to be registered by providing a valid email address and password. If the new account is registered correctly then the server should respond with AUTH REGISTERED, however, it should not log the user in. 

If the registration is unsuccessful the server should return one of the following ERROR messages:

\begin{itemize}

\item{INVALID_EMAIL \\
The email address was invalid}

\item{EMAIL_ALREADY_IN_USE \\
The email address has already by registered}

\item{PASSWORD_TOO_SHORT \\
The password is too short}

\item{MISSING_ARGUMENTS	\\
There were too few arguments}

\end{itemize}

Examples:
\texttt{:AUTH REGISTER: cyblob@gmail.com p455w0rd;} \\
\texttt{:AUTH REGISTER: blah lol;} \\
Server responds \texttt{:ERROR INVALID_EMAIL:} \\
\texttt{:AUTH LOGIN: cyblob@gmail.com p455w0rd;}\\
Server responds \texttt{:AUTH OKAY:} \\

{\bf :QUIT:;}

The \texttt{QUIT} command tells the server that the users wishes to log out (if applicable) and disconnect from the server. Once the quit command has been received the users’ state should be changed to OFFLINE and the the connection broken.

Examples:
\texttt{:QUIT:;} \\
User is logged out and disconnected.

\subsection{Client Commands}

{\bf :OKAY:;}

In the event that a command received from the client does not have any other response, the :OKAY: command is sent to signify that the command was successful. 

Examples:
\texttt{:OKAY:;}\\
The last command executed okay.

{\bf :SERVERSTATUS \{ USERS | TIME | UPTIME \}:;}

If requested by the client, the server may send data about the server status, as defined in section 4.1. If more than 1 argument is provided then the data is returned with each segment on its own line, in the same order which the arguments appeared. If no arguments are provided then the data can be in any format containing any information, so do not rely on it to have a consistent format.

Example:
\texttt{:SERVERSTATUS TIME: Wed 17 Nov, 23:31;}
\texttt{:SERVERSTATUS USERS: 56 Online, 423 Total;}
\texttt{:SERVERSTATUS UPTIME: 21 days, 14:41;}

{\bf :KILL: <message>;}

In the event that the server wants the client to disconnect, it will issue a \texttt{KILL} command. As soon as the command has been sent the server will disconnect. 

Example:
\texttt{:KILL: Too many bad login attempts.;}\\
The server closes the connection.

{\bf :BROADCAST: <message>;}

A BROADCAST message sent by the server to all users. The broadcast message should be displayed immediately and is likely to contain critical information about the server.

Example:
\texttt{:BROADCAST: The server is on fire and about to crash. ;}

{\bf :AUTH [ LOGGEDIN | UNAUTHORIZED ]:;}

The AUTH command indicated to the client its current status. If the user is logged in then it will have the LOGGEDIN argument, otherwise it will have the UNAUTHORISED argument.

Example:
\texttt{:AUTH LOGGEDIN:;}\\
The user is logged in and has permissions to use post-login commands

\section{Post-login Commands}

These commands are accessible only once the user has logged in. The server should issue an \texttt{:ERROR UNAUTHORISED:} response if an unauthorised user attempts to use one of these commands. After repeated attempts, the server may temporarily block or ban the user.

\subsection{Server Commands}

{\bf :SET [ NICKNAME| STATUS | PERSONAL_MESSAGE | DISPLAY_PIC ]: <value>;}

The SET commands allows various user attributes to be set by the client. The exact attribute depends on the augment given however only 1 attribute can be set at a time. In the case of DISPLAY_PIC, the image is Base 64 encoded.

Example:
\texttt{:SET NICKNAME: Andrew;}

{\bf :GET \{ NICKNAME| STATUS | PERSONAL_MESSAGE | DISPLAY_PIC \}: <user>\{<user>\};}

The GET command requests a set of attributes for each user in a space separated list of users. The server should respond with an INFO command and the appropriate data.

Example:
\texttt{:GET NICKNAME STATUS PERSONAL_MESSAGE DISPLAY_PIC: user1@host.com, user2@hotmail.com;}

{\bf :FRIENDLIST:;}

The FRIENDLIST command requests a list of users in the users friendlist. It should return a FREINDLIST command as specified in section 5.2

Example:
\texttt{:FREINDLIST:} 
	
{\bf :ROOM [ CREATE \{GROUP\} | INVITE | JOIN | LEAVE | USERS | TYPE ]: \{<roomid> | <user>\};}

The ROOM command deals with all room related requests.

\begin{itemize}

\item{CREATE \\
The CREATE arguments specifies that the client wishes to create a new chat room for use. If the GROUP argument is not supplied then the room will be limited to a maximum of 2 participants and a userID must be supplied so that the server can invite the other user. Upon creation of the room, the user who created it is automatically placed in the room.}

\item {INVITE \\
The INVITE argument sends an invite to join the room to the specified user. They are then allowed to join the room at any point.}

\item{JOIN \\
The JOIN command specifies that the user wishes to join the specified room. The must first have received an invite from someone already in the room.}

\item{LEAVE \\
The LEAVE argument specifies that the user wishes to leave the specified room.}

\item{USERS \\
The USERS argument request a list of the users in the room from the server separated by a space;}

\end{itemize}

\texttt{:MESSAGE: <roomid> <message>;}

The MESSAGE command specifies a room and a message which should be delivered to the room.

Example:
\texttt{:MESSAGE: 25 Hey guys, what’s up?;}

{\bf :FRIEND [ ADD | BLOCK | UNBLOCK | ACCEPT | DECLINE | DELETE ]: <target>;}

The FRIEND command controls all friend list data.

\begin{itemize}

\item{ADD \\
The ADD arguments specifies that the client wishes to add the target user to their friend list. The server should then send a request to the target user asking for permission to access their data.}

\item{BLOCK \\
The BLOCK argument places the target user into a list of blocked users who cannot access any data about the user.}

\item{UNBLOCK \\
The UNBLOCK argument removes the target from the current users blocked list.}

\item{ACCEPT \\
The ACCEPT arguments specifies that the user is responding to a previous friend request, where the target is the user who sent the request. If the user accepts then the target user should be given permissions to use the users data.}

\item{DELETE \\
The DELETE  argument specifies that the target user should be deleted from the users friend list, however the target user will still have access to the users details.}

\item{DECLINE \\
The DECLINE argument specifies that the user does not wish the target user to be to able to access their data.}

\end{itemize}

Example:
\texttt{:FRIEND ADD: cyblob@gmail.com;}	# Request access to cyblob@gmail.com’s data

{\bf :LOGOUT:;}

The LOGOUT command specifies that the client wishes to logout but not drop the connection to the server.

Example:
\texttt{:LOGOUT:;}\\
The user is logged out

\subsection{Client Commands}

{\bf :MESSAGE: <roomid> <sender> <message>;}

The \texttt{MESSAGE} command represents a message received by the user. The sender is the email address of the person who sent the message, and message is the message itself.

Example:
\texttt{:MESSAGE: 56 cyblob@gmail.com How’s it going?;}

{\bf :ROOM [ CREATED | JOINED | LEFT | INVITED | USERS | PERSONAL | GROUP ]: <roomid> \{<user>\};}

\begin{itemize}

\item{CREATED \\
The CREATED argument tells the client what their request to create a room was successful and specifies the room id.}

\item{JOINED \\
The JOINED argument tells the client that a new user has joined the room and specifies the room and id of the user who joined.}

\item{LEFT \\
The LEFT argument tells the client that a user has left the room and specifies the room and the id of the user who left.}

\item{INVITED \\
The INVITED argument tells that client that they have been invited to join a room. It specifies the room and the user who invited them;}

\item{USERS \\
The USERS argument provides a list of users (one per line) who are currently in the room.}

\item{PERSONAL \\
The PERSONAL argument tells the client that the room is a personal room (No more than 2 users).}

\item{GROUP \\
The GROUP argument tells the client that the room is a group room any any number of users can join.}
	
\end{itemize}

Example:
\texttt{:ROOM CREATED: 0;}
\texttt{:ROOM JOINED: 0 CYBLOB@GMAIL.COM;}
\texttt{:ROOM LEFT: 0 CYBLOB@GMAIL.COM;}
\texttt{:ROOM INVITED: 0 HAPPY0@GMAIL.COM; #HAPPY0@HOTMAIL.COM INVITED YOU TO THE ROOM;}
\texttt{:ROOM USERS: CYBLOB@GMAIL.COM HAPPY0@HOTMAIL.COM SOMEONEELSE@GMAAIL.COM ANOTHER@HOTMAIL.COM;}

{\bf :FRIENDLIST: ONLINE <user> <user>... OFFLINE <user> <user>... BLOCKED <user> <user>...;}

The FRIENDLIST command specifies the users friend list using the format specified above.

Example:
\texttt{:FREINDLIST: ONLINE cyblob@gmail.com blah@ddfdf.com OFFLINE meow@hotmail.com woof@gmail.com sheep@yahoo.com BLOCKED wtf@wtf.com;}

{\bf :FRIENDREQUEST: <user> <nickname>;}

The FRIENDREQUEST command tells the client that the user has a friend request and specifies who sent it and their nickname at the time of sending. The client should the respond with the appropriate FRIEND command.

Example:
\texttt{:FRIENDREQUEST: cyblob@gmail.com James;}

{\bf :UPDATE [ NICKNAME| STATUS | PERSONAL_MESSAGE | DISPLAY_PIC | FRIENEDLIST  ]: <user>;}

The UPDATE command notifies the client that a user in their friend list has just updated one of their attributes. The client can then retrieve the updated attribute if necessary.

Example:
\texttt{:UPDATE NICKNAME: cyblob@gmail.com;}

\texttt{:INFO \{ NICKNAME | STATUS | PERSONAL_MESSAGE | DISPLAY_PIC \}: <user> <data> \{<user> <data>\};} 

The data is returned one line per value, in the same order as the arguments given, separated by the users email address. In the case of \texttt{DISPLAY_PIC}, the image is Base 64 encoded to enable it to be embedded as text.

Example:

\texttt{:INFO NICKNAME STATUS PERSONAL_MESSAGE DISPLAY_PIC: user1@host.com
James
Away
I’m a panda
/9j/4AAQSkZJRgABAQAAAQABAAD/2wBDAAYEBAQFBAYFBQYJBgUGCQsIBgYICwwKCgsK...
user2@host.com
Gordon
Online
I’m not a panda
DAREAAhEBAxEB/8QAHQAAAgIDAQEBAAAAAAAAAAAABAUDBgECBwAICf/EAE4QAAIBAg...;

:ERROR [ UNAUTHORISED | INVALID_EMAIL | EMAIL_ALREADY_IN_USE | PASSWORD_TOO_SHORT | MISSING_ARGUMENTS | TOO_MANY_ARGUMENTS | INVALID_ARGUMENT | LOGGED_IN_FROM_OTHER_LOCATION | USER_DOES_NOT_EXIST | LOGIN_DETAILS_INCORRECT ]: <message>;}

If an error occurs then the server should return one of these commands to the client. The exact command and the message will be specified by the sever.

Example:
\texttt{:ERROR INVALID_EMAIL: No domain specified.;}


\chapter{Use Cases}
\label{use_cases}
\section{Introduction}

This appendix contains all of the use cases that were documented during the design process. In many cases, the specifics of how these actions occur in GIM are different than the outlines here. The higher priority use cases were documented while those less likely to be implemented were not.

\section{Actors}

Two actors were identified for the system: ``Senders'' and ``Contacts.'' Both of these actors have equal privileges within the system and use the system in the same way, hence both derive from the ``User'' type.

\section{Must Have}

\subsection{Open application}

{\bf Actors}: User

User opens the application. They are prompted to either login or register. The user must choose File-$>$Quit from the menu at the top to exit, as the X will only minimize the program.

\subsection{Register}

{\bf Actors}: User

User chooses to register, and then enters their email, nickname, and password (same window). If no one with that email or nickname has registered yet, the user is successfully registered, and can now log in. There is an option to login automatically, and it will do so if selected. If it isn't selected, they will be led back to the original screen (register or log-in).

\subsection{Log In}

{\bf Actors}: User

It is automatic, unless the option isn't checked. Otherwise, the user either knows their login info or chooses to reset their password (via email). They enter their email and password into the fields. If it is not correct, they are prompted again. When logged in, anyone who has them as a friend is shown that they are online.

\subsection{Add contact}

{\bf Actors}: Sender, Contact

User clicks the ``add contact'' button in the main window, or the option from the top menu. They are prompted to enter their friend's email. They enter the email and the system searches for that person in the database. If found, the contact is sent a request for friendship. If the contact accepts, then both the user and contact now have each other as contacts, and they show up in the main window. If not accepted, neither is added. If the contact is not found, they are prompted again to enter the email, with the option to cancel.

\subsection{Remove contact}

{\bf Actors}: User

User single-clicks the contact in their list that they want to remove and then presses the ``remove contact'' button in the main window (this can also be done from the top menu). This button will be grayed out until the user clicks a contact. A dialog box will pop up to confirm that they want to remove that person as a contact. There is also an option to block the contact. The contact will not have the user removed from their contact list.

\subsection{Chat/Send/Receive Messages}

{\bf Actors}: Sender, Contact

Four ways to begin a chat: selecting a contact and pressing the ``Enter'' key, double clicking a contact, right clicking a contact and selecting ``Chat,'' and selecting a contact then pressing a chat icon on the Client UI. The button to chat will be grayed out until the user clicks a contact.
A new window appears. The user types a message into the text box and clicks ``send'' or hits enter. This makes a new window show up on the contact's screen. Both the contact and the user see the message in the window. The user has the option of closing the window. If they reopen it, it will be cleared, unless they have the option chosen to show the last 10 or so messages with that contact.

\subsection{Change status}

{\bf Actors}: User

User clicks on their status and is given an option of either Online, Busy, Away, or Appear Offline. They click an option and their status is now shown as whichever one was picked. There will be some sort of icon to indicate a contact's status, and text shading will be used to indicate Online or Offline, possibly for other statues as well. Both the user and anyone who has the user as a contact (and is online) can see the changed status.

\subsection{Change nickname}

{\bf Actors}: User

User either clicks ``Options'' on the top menu and the options panel appears, or directly on their name in the main window and they type the new name there. When ``OK'' is clicked at the bottom of the window, the window disappears and the user, plus anyone who is online and has them as a friend, can see their changed nickname.

\subsection{Option change}

{\bf Actors}: User

User clicks ``Options'' on the top menu and the options window appears. User changes the options they want and clicks ``OK'' the changes are now made. \\
Changeable Options (in progress):\\
An option will be available to show the last 10 messages from a previous chat session with a contact when a new chat window is opened.

\subsection{Log Out}

{\bf Actors}: User

User clicks the ``log out'' option in the menu. The X in the corner will only minimize the program. A dialog box appears, asking if they are sure they want to log out. If yes, they are logged out, and the log in/register pop-up appears. If they want to quit the program completely, they will have to choose ``quit'' from the menu.

\section{Should Have}

\subsection{Add/remove/change Display Picture}

{\bf Actors}: User

User can click the display pic in the main window to bring up the options to change it, or go into the options menu.

\subsection{Send file}

{\bf Actors}: Sender, Contact

The user can right-click on a contact's name to show a context menu that includes ``Send file'' They can also choose to send the file from the chat window.

\subsection{Edit personal message}

{\bf Actors}: User

User will change this by clicking on the message itself and typing right there, or changing it in the options panel.

\subsection{Group Chat}

{\bf Actors}: Sender, Contact

User can choose to group chat from the main window button, main menu, or in a chat window with another contact. A window will show up in which the user can create a chatroom, and then the user will be prompted to choose the contacts they wish to invite. The chatroom appears to the user. Those contacts will see the invitation, and if they accept, will be placed in the chatroom. Contacts will be able to invite other people as well. The chatroom will disappear once every user has left.


\chapter{Organisation}
\label{Organisation}

\section{Schedule}

As with any software project, planning is an important aspect. As is to be expected from a group without much prior experience in team work, the original schedule was overly optimistic. Other course deadlines and a failure to appriciate the time requirements of each stage resulted in a failure to meet our planned deadlines. 

\begin{figure}[!h]
    \begin{center}
        \includegraphics[width=14cm]{appendix/Diagrams/GIMplan.png}
        \caption{Gantt chart of our planned schedule.}
        \label{lockingDia}
    \end{center}
\end{figure}

\begin{figure}[!h]
    \begin{center}
        \includegraphics[width=14cm]{appendix/Diagrams/GIMreal.png}
        \caption{Gantt chart of our actual schedule.}
        \label{lockingDia}
    \end{center}
\end{figure}

The process of designing the system took longer than was initially anticipated, which delayed the implementation process significantly. The implementation process for basic functionality went smoothly and was finished relatively quickly, however a large amount of time was spent improving the interface and polishing the program. Our original plan provided almost 2 months to write this dissertation, in reality we had only 1 month.  

\section{Version Control}

For this project we used version control to manage code and documentation. All internal documentation was hosted using Google Docs, while this document and the java code was managed using SVN provided by Google Code. Our project initally used the SVN provided by the Computing Science school in Glasgow University, however, due to stability and relability issues we moved our project to Google's infrastructure.


\chapter{Status Report}
\documentclass{article}
\begin{document}

\section{Status}

\label{appenstatus}

The program is stable and does not appear to have any substantial bugs. The program meets the requirements outlined in the problem definition. \\

{\bf List of known issues with GIM:}
\begin{itemize}

\item{In some instances, a contact will come online and cause the users' contact list to stop displaying contacts. The user can fix this by selecting the "Display Offline Users" option.}

\item{A long line of text will necessitate the user to scroll across the chat area, as oposed to the text wrapping round.}

\item{The Block feature acts more like an ignore feature; the user that blocks a contact still appears online that contact.}

\item{If a user goes offline while talking to a contact, any messages that the contact sends while the user is offline will be send when they sign back online, assuming the contact remains online. This could be considered a feature of the system, as this it is a partial implementation of offline messages. As the user is not made aware that this will happen, the issue is considered a defect.}

\end{itemize}

\end{document}


\chapter{Contributions}
\section{Summary Log}
\label{sumlog}

James:

Organisation: Version Control, Server Hosting
Design: Server design, protocol specification architect
Implementation: Server, GUI Friend List JList renderer. GUI User data listeners.
Report: Protocol Design, Server Design + Implementation

Heather:

Organisation: Quality Assurance, report editor. Planning. Internal meetings manager.
Design: Interface design, requirements analysis, Client Structure
Implementation: GUI Front end, some networking interface implementation
Report: Introduction, Problem Definition, Conclusion. GUI Implementation section. Section introductions, section summaries. Editing. 

Ewan:

Design: Interface design, requirements analysis
Documentation: Use cases, wire frames
Implementation: GUI Front end
Report: Design of GUI, Requirements (features), Appendix 

Gordon:

Organisation: Team secretary
Design: Client Structure, some contributions to Protocol Specification
Implementation: Network Component, Gui Backend (model and controller)
Report: Client Design, Networking Design, (Client Implementation), Evaluation



%==============================================================================
\bibliographystyle{plain}
\bibliography{example}
\end{document}
