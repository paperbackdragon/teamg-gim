\section{User Evaluation}
\label{user_eval}

As the primary focus of this project was on networking, we did not formally evaluate GIM with users. User evaluations would only serve to evaluate the interface, which, with time constraints, was not a priority. Within the team, we felt the interface was fit for purpose and any data gained from such an evaluation would have little practical use.

Informal evaluations occurred from the point when the server had basic functionality. Other Computing Science students volunteered to use the program throughout implementation, starting before any interface was in place. This process was used for testing, as more traffic though the server increased the chance of finding bugs or anomalies. An example of where this helped was in routing messages to their respective chat rooms; messages were not being routed correctly and user testing allowed us to track down the cause of the issue.

There are also advantages in unstructured testing; Computing Science students will typically try to abuse and find defects with any program. For example, the use of HTML code working in chat windows was discovered in this way, and subsequently fixed.
