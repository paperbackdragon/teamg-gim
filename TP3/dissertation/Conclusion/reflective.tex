%TODO: add a short paragraph summarizing the paper.

Over the course of this project, we've all learned a lot as a team. The importance of communication and organisation, the design decisions required of a large project, and how to split up work fairly. We've also learned more about networking and making GUIs in Java Swing. The following is a more in-depth discussion of how the project went for us.

Early on, we had a sort of `laissez-faire' style of team management, with no one in particular in charge. We had few strict roles among us either, though a few had been decided. Gordon was willing to be secretary and take meeting minutes, and Heather was experienced with writing and editing, as well as being good at organisation, so she took the role of Quality Assurance. While we were designing the project though, we didn't have any other defined roles.

As we progressed further into the design, it was clear we needed to split up the work in a manageable way if we wanted things to be fair. We decided to have Gordon and James work on the server and Ewan and Heather to work on the client and corresponding GUI. This worked well, since James and Gordon had more of an interest in networking, and Ewan and Heather were more experienced with making GUIs. Soon we found we hadn't made a clear line between the server and client work, and thus some of the networking code was split between the two smaller teams. The client team would have had a lot more work than the server team, considering how much networking code it involved. We then came up with the idea of the two interfaces (See section \ref{designimpl}), which helped more precisely define the work boundary between the two teams. The server team implemented methods that the client would call, and the client team implemented methods that the server would call. This more fairly split up the work and made it manageable.

Over the first term, we met weekly as a team, in addition to our meetings with our supervisor. These supervisor meetings in particular helped us stay on track, since we went over what we did each week and then discussed our plans for the following weeks. The meetings also gave us formative feedback about our approach to the project and the dissertation. We likely would have been more disorganised without them. The team meetings were helpful too, as we all focused at the same time, in person, on what had to be done that week.

Then when the long winter break came, we decided to set up an IRC channel for ourselves to help communicate, especially since both Heather and James traveled and weren't able to meet with the rest of the team in person. This IRC channel helped us immensely, and we all had enough interest in the project to get a fair amount of coding done over the break. Once the next term had started, we had the client and server communicating mostly successfully. Over this term, we didn't have the same convenient hour between lectures to meet, and started to rely mostly on IRC for communication. In retrospect we probably would have benefited from continuing to meet every week, since at this point not all the team members were aware of what aspects of the project needed to be done. The use of tools like SVN,\footnote{\texttt{http://subversion.apache.org/}} though, were helpful in that regard, as we were able to read the commit logs and see what other members had recently done. Most of the work had been finished over the break anyway, and at this point there were mostly just little bugs to be fixed.

Laissez-faire management models don't always get good results, since some types of people need more direction, someone telling them what to do. Over the course of the year, we observed other teams struggling as some of their team members did little or no work towards their projects, even teams without the laissez-faire model. But we were lucky in that regard, as we were a team of individuals that were all ready and willing to learn and work with our team.

We've all learned a lot about dealing with large projects, as we have all never worked on a project of this size. Aside from team management, the amount of designing and coding in general was daunting at times. The fact that we were able to split up the work into teams of people who were good at what they were working on was very beneficial. In addition, coordinating how each component communicated and worked with one another--the GUI to the client model, for example--was a challenging part of this project. To overcome this, we spent time working out the confusion with each other, and ended up with a design that worked decently for the scope of this project.

Splitting up the work into teams had its downsides, though. As James and Gordon were focusing on the server and how it communicated with the client, they didn't get as much experience with GUIs and the HCI aspects of the project. Also, vice versa, Ewan and Heather didn't get as intimate knowledge of the workings of the server and its networking aspects as they would if they had been given tasks working with the server. We did have to learn what each other was working on in order to communicate, but we didn't learn as much about each other's work as we might have if we'd all had a hand in each part of the project. Despite this lack of learning, overall it did help the project progress faster since everyone was working on what they knew about.

Overall, we think this project has been a success. Maybe each of us separately hasn't learned every detail about how to create an instant messenger from scratch, but collectively we certainly have, and we've learned a lot about team and large project management in the process. Our team worked well together and communicated effectively enough, despite our slow disorganisation at the end. We hope that GIM will be used in some way in the future, perhaps studied by others interested in this topic, extended by one of us in a future project, or even used to do what it was made for: instant messaging.

