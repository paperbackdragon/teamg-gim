\section{Server Design}

The GIM server is primarily responsible for synchronising the communication between the connected clients and keeping a persistent record of user information. In particular the server is responsible for:

\begin{itemize}
    \item{Authenticating users}
    \item{Routing chat messages for one client to one or more other clients}
    \item{Sending requests from one client to another client (such as friend requests or chat invites)}
    \item{Notifying subscribed users of changes to another user's information}
    \item{Storing user information such as nicknames, passwords, display pictures, friend lists, etc.}
\end{itemize}

\subsection{Server Structure}

From the very beginning the server was designed to be secure, scalable, and simple. Users must trust it to keep their information safe and route messages correctly, and it must be able to cope with any number of connected clients.

In order to manage this, the server generates a new worker for every connected client. This new worker acts as the single point of contact for its respective client. This greatly simplifies the design of the server as we can treat each connection individually and allows the server to easily scale to a large number of clients.

\begin{figure}[!h]
    \begin{center}
        \includegraphics[scale=0.6]{Design/diagrams/server_workers.png}
        \caption{The GIM Server with 3 connected clients, each with their own worker.}
        \label{WorkersDia}
    \end{center}
\end{figure}

Each worker has 2 buffers, one which stores commands read from the network (the command buffer), and one which stores commands to be written to the network (the response buffer). The worker continually reads commands from the command buffer, processes them, and puts the responses into the response buffer. At the same time the worker is also performing network reads and writes to fill and empty the respective buffers.

The buffers allow the worker to run using several threads which increases the performance on multi-core systems as the server is able to read from the network, process commands, and write to the network in a truly concurrent fashion. This means that at any one time a worker could be handling up to 3 different commands, each at a different stage of execution, and still have several commands stored in the buffers waiting to be used.

\begin{figure}[!h]
    \begin{center}
        \includegraphics[scale=0.6]{Design/diagrams/worker_detail.png}
        \caption{The internal components of a single worker.}
        \label{WorkerDatailedDia}
    \end{center}
\end{figure}

\subsection{Client-to-Client Messaging}
\label{c2c}
No direct client to client communication is possible, however workers are able to indirectly communicate with other clients (i.e. clients to which they are not connected) by passing commands to the other client's worker. This is the server's main method of inter-client communication and relies heavily on the use of the buffers to pass messages from one worker to the network writer or command processor of another worker.

As the protocol defines no client to client communication is possible before a user has logged in, and as user can only be logged in on one client at a time, only a user id is necessary to send a command to a user. As long as a record is kept of which worker belongs to each online user, then sending a message to a user is no more complicated than identifying the particular user.

\subsection{Message Routing}
\label{message_routing}

Message routing in the server is fairly simple because of the use of Rooms. A Room is a collection of one or more users and each Room has a unique identifier. In order to join a Room a user must either create a new Room (in which case they are automatically placed into the Room) or they must be invited to the Room by a user already in the Room. A Room can be one of two types: Personal or Group. A Personal Room is much more restrictive than a Group Room. A Personal Room must be provided with the identifier of one other user at time of creation and is limited to only 2 users. Once a Personal Room has been created you cannot invite any other users. A Group Room has no such restrictions. Any number of users may be invited to a Group Room at any time.

All chat messages in GIM are sent to Rooms rather than particular users. The server must maintain a list of the users in each room, and open receiving a message for that room, forward it to all of the users in the room (excluding the sender of the message). This is done using the method described in section \ref{c2c}. A \texttt{:MESSAGE:} command is generated and placed into the Response Buffer of the worker assigned to each of the recipients ready for the worker to send to the client.

\subsection{Subscriptions and Notifications}
One of the servers most critical responsibilities is notifying users of changes to the users they are subscribed to. A user is considered to be subscribed to users who are in their friend list or both users are currently in the same Room.

\subsection{User Accounts}
A user is a registered account on the server with a unique identifier (an email address), password, nickname, personal message, display picture, status, and friend list. 

\subsubsection{Identification}
The user's identifier and password are used to authenticate the user at login. These cannot be changed one the account has been created. 

\subsubsection{Nickname}
A user's nickname is a short and familiar name for that user. A nickname does not need to be unique and is not intended to be used as an identifier by the system. By default a user's nickname is their email address.

\subsubsection{Personal Message}
The personal message is a medium length message (usually only a sentence or two) which the user can use to share news or any information about themselves.

\subsubsection{Display Picture}
A Display Picture is a small image chosen by the user, often of themselves, so that they can be easily identified visually. By default a user's display picture is the GIM logo. 

\subsubsection{Status}
A user can have one of several statuses: Online, Away, Busy and Appear Offline. The first three statuses are used simply as an indication of the user's likelihood to respond to a message. The Appear Offline status allows the user to appear to other users as though they were not signed in but still receive updates and messages from other users.

\subsubsection{Friend List}
Each user has their own Friend List, a list of users who have accepted a friend request from the owner of the Friend List. When a user accepts a friend request, the sender and receiver are added to each other's Friend List. The server only grants access to information about a user if that user is in the requesting user's Friend List or if the users are currently in the same Room. This means that simply removing a user from your Friend List does not stop them from being able to view your information or message you.

A user is considered to be `subscribed' to all of the users in their Friend List. This means that they receive update notifications from the server whenever a user in their Friend List updates their nickname, personal message, display picture or status.

\subsubsection{Blocked User}
A user is able to stop a another user from communicating with them and viewing their information by blocking that user. It is possible to block any user, even users not currently in your friend list. Blocking a user is a one-way process and does not stop a user from messaging or viewing information about users they have blocked.

\subsection{Synchronisation}

The server has a large section of backing data which stores all of the users, rooms, workers and any other operational data required by the server. Due to the highly concurrent nature of the server it is very important to ensure that the data is properly synchronised. This is to prevent race conditions where two writes occur on an object at the same time giving an unpredictable output. 

\begin{figure}[!h]
    \begin{center}
        \includegraphics[scale=0.6]{Design/diagrams/server_race_condition.png}
        \caption{Unsynchronised workers updating the same object at the same time, resulting in a race condition and giving the unexpected value of 6 rather than 7.}
        \label{RaceConditionDia}
    \end{center}
\end{figure}

To stop this the server must lock (claim ownership of) the data before it is updated. This is a very tricky thing to do as as the lock must be long enough to ensure that no updates occur between reading the original data and writing the updated data, whilst also keeping the lock as short as possible to avoid long periods of blocking and keep performance high.

\begin{figure}[!h]
    \begin{center}
        \includegraphics[scale=0.6]{Design/diagrams/server_locking.png}
        \caption{Synchronised workers using locking to read and write the updated value, giving the expected value of 7.}
        \label{lockingDia}
    \end{center}
\end{figure}

This means that the data must be carefully managed...blah

\newpage
\begin{landscape}
    \begin{figure}
        \begin{center}
            \includegraphics[scale=0.5]{Design/diagrams/server_uml.png}
            \caption{UML showing the structure of the GIM Server}
            \label{highLevelDia}
        \end{center}
    \end{figure}
\end{landscape}
